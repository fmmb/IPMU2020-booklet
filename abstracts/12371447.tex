\documentclass[../booklet.tex]{subfiles}

\begin{document}

\section[Is the invariance with respect to powers of a t-norm a restrictive property on fuzzy implication functions? The case of strict t-norms. {\it Raquel Fernandez-Peralta, Sebastia Massanet and Arnau Mir}]{Is the invariance with respect to powers of a t-norm a restrictive property on fuzzy implication functions? The case of strict t-norms}
\index[authors]{Fernandez-Peralta, Raquel} \index[authors]{Massanet,  Sebastia} \index[authors]{Mir, Arnau}

\begin{center}
  {\it Raquel Fernandez-Peralta, Sebastia Massanet and Arnau Mir}
\end{center}
%\begin{minipage}{1\textwidth}
%\end{minipage}

\vskip 0.8cm


The invariance with respect to powers of a t-norm has emerged as an important property for fuzzy implication functions in approximate reasoning. Recently, those fuzzy implication functions satisfying this property where fully characterized leading to seemingly new families of these operators. In this paper, the additional properties of the family of fuzzy implication functions which are invariant with respect to powers of a strict t-norm are analyzed. In particular, properties such as the exchange principle, the law of importation with respect to a t-norm or the left neutrality principle, among others, can be fulfilled by some members of this family. This study allows to characterize the intersection of these operators with the most important families of fuzzy implication functions.  

\keywords{Fuzzy implication function  \and Invariance \and powers of t-norms \and Exchange principle.}



\end{document}
