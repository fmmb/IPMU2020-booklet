\documentclass[../booklet.tex]{subfiles}

\begin{document}

\subsection[Towards a logic-based view of some approaches to classification tasks. {\it Didier Dubois and Henri Prade}]{Towards a logic-based view of some approaches to classification tasks}
\index[authors]{Dubois, Didier} \index[authors]{Prade, Henri}

\begin{center}
  {\it Didier Dubois and Henri Prade}
\end{center}
%\begin{minipage}{1\textwidth}
%\end{minipage}

%\vskip 0.8cm


This  paper is a plea for revisiting various existing approaches to the handling of data, for classification purposes, based on a set-theoretic view, such as version space learning, formal concept analysis, or analogical proportion-based inference, which rely on different paradigms and motivations and have been developed separately. The paper also exploits the notion of conditional object as a proper tool for 
modeling if-then rules. It also advocates possibility theory for handling uncertainty in such settings. 
It is a first, and preliminary, step towards a unified view of what these approaches contribute to machine learning. \\

\keywords{data, classification, version space, conditional object, if-then rule, analogical proportion, formal concept analysis, possibility theory, possibilistic logic, bipolarity, uncertainty}



\end{document}
