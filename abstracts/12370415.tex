\documentclass[../booklet.tex]{subfiles}

\begin{document}

\subsection[Possibilistic Estimation of Distributions to Leverage Sparse Data in Machine Learning. {\it Andrea Tettamanzi, David Emsellem, Célia Da Costa Pereira, Alessandro Venerandi and Giovanni Fusco}]{Possibilistic Estimation of Distributions to Leverage Sparse Data in Machine Learning}
\index[authors]{Tettamanzi, Andrea} \index[authors]{Emsellem, David} \index[authors]{Pereira, Célia Da Costa} \index[authors]{Venerandi, Alessandro} \index[authors]{Fusco, Giovanni}

\begin{center}
  {\it Andrea Tettamanzi, David Emsellem, Célia Da Costa Pereira, Alessandro Venerandi and Giovanni Fusco}
\end{center}
%\begin{minipage}{1\textwidth}
%\end{minipage}

\vskip 0.8cm


Prompted by an application in the area of human geography using machine learning to
study housing market valuation based on the urban form, we propose a method based on
possibility theory to deal with sparse data, which can be combined with any machine learning
method to approach weakly supervised learning problems.
More specifically, the solution we propose constructs a possibilistic
loss function to account for an uncertain supervisory signal.
Although the proposal is illustrated on a specific application, its basic principles
are general. The proposed method is then empirically validated on real-world data.



\end{document}
