\documentclass[../booklet.tex]{subfiles}

\begin{document}

\section[On Solutions of Marginal Problem in Evidence Theory. {\it Jirina Vejnarova}]{On Solutions of Marginal Problem in Evidence Theory}
\index[authors]{Vejnarova, Jirina}

\begin{center}
  {\it Jirina Vejnarova}
\end{center}
%\begin{minipage}{1\textwidth}
%\end{minipage}

\vskip 0.8cm


Recently introduced marginal problem -- which addresses the question
of whether or not a common extension exists for a given set of
marginal basic assignments -- in the framework of evidence theory is
recalled. Sets of solutions are studied in more detail and it is
shown, by a simple example, that their structure is much more
complicated (i.e. the number of extreme vertices of the convex set
of solutions is substantially greater) than that in an analogous
problem in probabilistic framework. The concept of product extension
of two basic assignments is generalized (via operator of
composition) to a finite sequence of basic assignments. This makes
possible not only to express the extension, if it exists, in a
closed form, but also enables us to find the sufficient condition
for the existence of an extension of evidential marginal problem.
Presented approach is illustrated by a simple example.

\keywords{Marginal problem  \and Extension \and Product extension.}



\end{document}
