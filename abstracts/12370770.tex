\documentclass[../booklet.tex]{subfiles}

\begin{document}

\subsection[Imprecise Classification with Non-Parametric Predictive Inference. {\it Serafín Moral-García, Carlos J. Mantas, Javier G. Castellano and Joaquin Abellán}]{Imprecise Classification with Non-Parametric Predictive Inference}
\index[authors]{Moral-García, Serafín} \index[authors]{Mantas, Carlos J.} \index[authors]{Castellano, Javier G.} \index[authors]{Abellán, Joaquin}

\begin{center}
  {\it Serafín Moral-García, Carlos J. Mantas, Javier G. Castellano and Joaquin Abellán}
\end{center}
%\begin{minipage}{1\textwidth}
%\end{minipage}

%\vskip 0.8cm



In many situations, classifiers predict a set of states of a class variable because there is no information enough to point only one state. In the data mining area, this task is known as Imprecise Classification. Decision Trees that use imprecise probabilities, also known as Credal Decision Trees (CDTs), have been adapted to this field. The adaptation proposed so far uses the Imprecise Dirichlet Model (IDM), a mathematical model of imprecise probabilities that assumes prior knowledge about the data, depending strongly on a hyperparameter. This strong dependence is solved with the Non-Parametric Predictive Inference Model (NPI-M), also based on imprecise probabilities. This model does not make any prior assumption of the data and does not have parameters. In this work, we propose a new adaptation of CDTs to Imprecise Classification based on the NPI-M. An experimental study carried out in this research shows that the adaptation with NPI-M has an equivalent performance than the one obtained with the adaptation based on the IDM with the best choice of the hyperparameter.  Consequently, since the NPI-M is a non-parametric approach, it is concluded that the NPI-M is more appropriated than the IDM to be applied to the adaptation of CDTs to Imprecise Classification. 

\keywords{Imprecise Classification, Credal Decision Trees, IDM, NPI-M, imprece probabilities.}



\end{document}
