\documentclass[../booklet.tex]{subfiles}

\begin{document}

\subsection[SK-MOEFS: A Library in Python for Designing Accurate and Explainable Fuzzy Models. {\it Gionatan Gallo, Vincenzo Ferrari, Francesco Marcelloni and Pietro Ducange}]{SK-MOEFS: A Library in Python for Designing Accurate and Explainable Fuzzy Models}
\index[authors]{Gallo, Gionatan} \index[authors]{Ferrari, Vincenzo} \index[authors]{Marcelloni, Francesco} \index[authors]{Ducange, Pietro}

\begin{center}
  {\it Gionatan Gallo, Vincenzo Ferrari, Francesco Marcelloni and Pietro Ducange}
\end{center}
%\begin{minipage}{1\textwidth}
%\end{minipage}

\vskip 0.8cm


Recently, the explainability of Artificial Intelligence (AI) models and algorithms is becoming an important requirement in real-world applications. Indeed, although AI allows us to address and solve very difficult and complicated problems, AI-based tools act as a black box and, usually, do not explain how/why/when a specific decision has been taken. Among AI models, Fuzzy Rule-Based Systems (FRBSs) are recognized world-wide as transparent and interpretable tools: they can provide explanations in terms of linguistic rules. Moreover, FRBSs may achieve accuracy comparable to those achieved by less transparent models, such as neural networks and statistical models. In this work, we introduce SK-MOEFS (acronym of SciKit-Multi Objective Evolutionary Fuzzy System), a new Python library that allows the user to easily and quickly design FRBSs, employing Multi-Objective Evolutionary Algorithms. Indeed, a set of FRBSs, characterized by different trade-offs between their accuracy and their explainability, can be generated by SK-MOEFS. The user, then, will be able to select the most suitable model for his/her specific application.

\keywords{Explainable Artificial Intelligence \and Multi-objective Evolutionary Algorithms \and Fuzzy Rule-Based Systems  \and Python \and Scikit-Learn.}



\end{document}
