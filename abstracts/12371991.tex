\documentclass[../booklet.tex]{subfiles}

\begin{document}

\subsection[Group definition based on flow in community detection. {\it María Barroso, Inmaculada Gutiérrez, Daniel Gómez, Javier Castro and Rosa Espínola}]{Group definition based on flow in community detection}
\index[authors]{Barroso, María} \index[authors]{Gutiérrez,  Inmaculada} \index[authors]{Gómez,  Daniel} \index[authors]{Castro,  Javier} \index[authors]{Espínola, Rosa}

\begin{center}
  {\it María Barroso, Inmaculada Gutiérrez, Daniel Gómez, Javier Castro and Rosa Espínola}
\end{center}
%\begin{minipage}{1\textwidth}
%\end{minipage}

\vskip 0.8cm



Community detection problems are one of the hottest disciplines in social network analysis. Nevertheless, most of the related algorithms are specific for non-directed networks, or are based on a density concept of group. In this paper, we deal with a new concept of community for directed networks that is based on the classical flow concept.  A community is strong and cohesive if their members can communicate among them. With the aim of dealing with the identification of this new class of groups, in this work, we propose the use of fuzzy measures to represent the flow capacity of a group. We also provide a competitive community detection algorithm that focus on the identification of these new class of flow-based community. 
\color{black}

\keywords{Directed Networks  \and Flow \and Fuzzy Measures \and Community Detection Problem \and Louvain Algorithm.}



\end{document}
