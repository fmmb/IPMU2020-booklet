\documentclass[../booklet.tex]{subfiles}

\begin{document}

\subsection[On relevance of linguistic summaries - a case study from the agro-food domain. {\it Anna Wilbik, Diego Barreto and Ge Backus}]{On relevance of linguistic summaries - a case study from the agro-food domain}
\index[authors]{Wilbik, Anna} \index[authors]{Barreto, Diego} \index[authors]{Backus, Ge}

\begin{center}
  {\it Anna Wilbik, Diego Barreto and Ge Backus}
\end{center}
%\begin{minipage}{1\textwidth}
%\end{minipage}

\vskip 0.8cm


We present an application of linguistic summaries in the agro-food domain. We focus on the relevance aspect. Using the interviews we  determine which linguistic summaries are useful and appropriate for target users (farmers). The user evaluation with a TAM survey indicates that linguistic summaries allow farmers to understand quickly the past performance of their pig barns.

\keywords{Linguistic summaries \and Relevance \and Case study \and Computing with Words}



\end{document}
