\documentclass[../booklet.tex]{subfiles}

\begin{document}

\section[Automatic Truecasing of Video Subtitles using BERT: A multilingual adaptable approach. {\it Ricardo Rei, Nuno Miguel Guerreiro and Fernando Batista}]{Automatic Truecasing of Video Subtitles using BERT: A multilingual adaptable approach}
\index[authors]{Rei, Ricardo} \index[authors]{Guerreiro,  Nuno Miguel} \index[authors]{Batista, Fernando}

\begin{center}
  {\it Ricardo Rei, Nuno Miguel Guerreiro and Fernando Batista}
\end{center}
%\begin{minipage}{1\textwidth}
%\end{minipage}

\vskip 0.8cm

 
% The abstract should briefly summarize the contents of the paper in 15--250 words.
This paper describes an approach for automatic capitalization of text without case information, such as spoken transcripts of video subtitles, produced by automatic speech recognition systems. Our approach is based on pre-trained contextualized word embeddings, requires only a small portion of data for training when compared with traditional approaches, and is able to achieve state-of-the-art results. The paper reports experiments both on general written data from the European Parliament, and on video subtitles, revealing that the proposed approach is suitable for performing capitalization, not only in each one of the domains, but also in a cross-domain scenario. We have also created a versatile multilingual model, and the conducted experiments show that good results can be achieved both for monolingual and multilingual data. Finally, we applied domain adaptation by finetuning models, initially trained on general written data, on video subtitles, revealing gains over other approaches not only in performance but also in terms of computational cost.

\keywords{Automatic capitalization \and Automatic Truecasing \and BERT \and Contextualized Embeddings \and Domain Adaptation.}



\end{document}
