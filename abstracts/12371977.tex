\documentclass[../booklet.tex]{subfiles}

\begin{document}

\section[Improvements on the Convergence and Stability of Fuzzy Grey Cognitive Maps. {\it Istvan Harmati and Laszlo T. Koczy}]{Improvements on the Convergence and Stability of Fuzzy Grey Cognitive Maps}
\index[authors]{Harmati, Istvan} \index[authors]{Koczy, Laszlo T.}

\begin{center}
  {\it Istvan Harmati and Laszlo T. Koczy}
\end{center}
%\begin{minipage}{1\textwidth}
%\end{minipage}

\vskip 0.8cm


Fuzzy grey cognitive maps (FGCMs) are extensions of fuzzy cognitive maps (FCMs), where the causal connections between the concepts are represented by so-called grey numbers.  Just like in classical FCMs, the inference is determined by an iteration process, which may converge to an equilibrium point, but limit cycles or chaotic behaviour may also show up.

In this paper, based on network measures like in-degree, out-degree and connectivity, we provide new sufficient conditions for the existence and uniqueness of fixed points for FGCMs. Moreover, a tighter convergence condition is presented using the spectral radius of the modified weight matrix.

\keywords{Fuzzy cognitive map  \and Fuzzy grey cognitive map \and Stability \and Convergence \and Equilibrium point}



\end{document}
