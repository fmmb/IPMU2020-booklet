\documentclass[../booklet.tex]{subfiles}

\begin{document}

\section[Unification in Lukasiewicz logic with a finite number of variables. {\it Marco Abbadini, Federica Di Stefano and Luca Spada}]{Unification in Lukasiewicz logic with a finite number of variables}
\index[authors]{Abbadini, Marco} \index[authors]{Stefano,  Federica Di} \index[authors]{Spada, Luca}

\begin{center}
  {\it Marco Abbadini, Federica Di Stefano and Luca Spada}
\end{center}
%\begin{minipage}{1\textwidth}
%\end{minipage}

\vskip 0.8cm


We prove that the unification type of {\L}ukasiewicz logic with a finite number of variables is either infinitary or nullary.  To achieve this result we use Ghilardi's categorical characterisation of unification types in terms of projective objects,  the categorical duality between finitely presented MV-algebras and rational polyhedra, and a homotopy-theoretic argument.
\keywords{{\L}ukasiewicz logic \and MV-algebras \and unification \and universal cover}



\end{document}
