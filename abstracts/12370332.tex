\documentclass[../booklet.tex]{subfiles}

\begin{document}

\subsection[Acoustic feature selection with fuzzy clustering, self organizing maps and psychiatric assessments. {\it Olga Kamińska, Katarzyna Kaczmarek-Majer and Olgierd Hryniewicz}]{Acoustic feature selection with fuzzy clustering, self organizing maps and psychiatric assessments}
\index[authors]{Kamińska, Olga} \index[authors]{Kaczmarek-Majer, Katarzyna} \index[authors]{Hryniewicz, Olgierd}

\begin{center}
  {\it Olga Kamińska, Katarzyna Kaczmarek-Majer and Olgierd Hryniewicz}
\end{center}
%\begin{minipage}{1\textwidth}
%\end{minipage}

%\vskip 0.8cm




%Bipolar disorder is a serious mental illness characterized with changing episodes from depression to mania and mixed states.
%Hopefully, 
Acoustic features about phone calls are promising markers for prediction of bipolar disorder episodes. Smartphones enable collection of voice signal on a daily basis, and thus, the amount of data available for analysis is quickly growing. At the same time, even though the collected data are crisp, there is a lot of imprecision related to the extraction of acoustic features, as well as to the assessment of patients' mental state. In this paper, we address this problem and perform an advanced approach to feature selection. We start from the recursive feature elimination, then two alternative approaches to clustering (fuzzy clustering and self organizing maps) are performed.
%OK: dodałam "partially assumed labels"
Finally, taking advantage of the partially assumed labels about the state of a patient derived from psychiatric assessments, we calculate the degree of agreement between clusters and labels aiming at selection of most adequate subset of acoustic parameters. The proposed method is preliminary validated on the real-life data gathered from smartphones of bipolar disorder patients.

\keywords{Self organizing maps, Fuzzy C-Means, Recursive Feature Selection, Cluster Agreement, Bipolar Disorder Episode Prediction}



\end{document}
