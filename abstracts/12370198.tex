\documentclass[../booklet.tex]{subfiles}

\begin{document}

\section[Hierarchical reasoning and knapsack problem modelling to design the ideal assortment in retail. {\it Jocelyn Poncelet, Pierre-Antoine Jean, Jacky Montmain and Michel Vasquez}]{Hierarchical reasoning and knapsack problem modelling to design the ideal assortment in retail}
\index[authors]{Poncelet, Jocelyn} \index[authors]{Jean,  Pierre-Antoine} \index[authors]{Montmain,  Jacky} \index[authors]{Vasquez, Michel}

\begin{center}
  {\it Jocelyn Poncelet, Pierre-Antoine Jean, Jacky Montmain and Michel Vasquez}
\end{center}
%\begin{minipage}{1\textwidth}
%\end{minipage}

\vskip 0.8cm


The survival of a supermarket chain is heavily dependent on its capacity to maintain the loyalty of its customers. Proposing adequate products to customers is the issue of the store's assortment. With tens thousands of products on shelves, designing the ideal assortment is theoretically a thorny combinatorial optimization problem. The approach we propose includes prior knowledge on the hierarchical organization of products by family to formalize the ideal assortment problem into a knapsack problem. The main difficulty of the optimization problem remains the estimation of the expected benefits associated to changes in the product range of products' families. This estimate is based on the accounting results of similar stores. The definition of the similarity between two stores is then crucial. It is based on the prior knowledge on the hierarchical organization of products that allows approximate reasoning to compare any two stores and constitutes the major contribution of this paper.
\keywords{Optimal Assortment in Mass Distribution, Semantic Similarity Measures, Knapsack problem.}



\end{document}
