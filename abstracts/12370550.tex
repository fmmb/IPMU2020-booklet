\documentclass[../booklet.tex]{subfiles}

\begin{document}

\subsection[Performance and Interpretability in Fuzzy Logic Systems – can we have both?. {\it Direnc Pekaslan, Chao Chen, Christian Wagner and Jonathan M. Garibaldi}]{Performance and Interpretability in Fuzzy Logic Systems – can we have both?}
\index[authors]{Pekaslan, Direnc} \index[authors]{Chen, Chao} \index[authors]{Wagner, Christian} \index[authors]{Garibaldi, Jonathan M.}

\begin{center}
  {\it Direnc Pekaslan, Chao Chen, Christian Wagner and Jonathan M. Garibaldi}
\end{center}
%\begin{minipage}{1\textwidth}
%\end{minipage}

\vskip 0.8cm



Fuzzy Logic Systems can provide a good level of interpretability and may provide a key building block as part of a growing interest in explainable AI. In practice, the level of interpretability of a given fuzzy logic system is dependent on how well its key components, namely, its rule base and its antecedent and consequent fuzzy sets are understood. The latter poses an interesting problem from an optimisation point of view -- if we apply optimisation techniques to optimise the parameters of the fuzzy logic system, we may achieve better performance (e.g. prediction), however at the cost of poorer interpretability. In this paper, we build on recent work in non-singleton fuzzification which is designed to model noise and uncertainty `where it arises', limiting any optimisation impact to the fuzzification stage. We explore the potential of such systems to deliver good performance in varying-noise environments by contrasting one example framework - ADONiS, with ANFIS, a traditional optimisation approach designed to tune all fuzzy sets. Within the context of time series prediction, we contrast the behaviour and performance of both approaches with a view to inform future research aimed at developing fuzzy logic systems designed to deliver both -- high performance and high interpretability. 

\keywords{Non-Singleton Fuzzy System  \and Interpretability \and ADONiS \and ANFIS \and Parameter Tuning}



\end{document}
