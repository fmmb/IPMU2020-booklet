\documentclass[../booklet.tex]{subfiles}

\begin{document}

\subsection[A fuzzy model for interval-valued time series modeling and application in exchange rate forecasting. {\it Leandro Maciel, Rosangela Ballini and Fernando Gomide}]{A fuzzy model for interval-valued time series modeling and application in exchange rate forecasting}
\index[authors]{Maciel, Leandro} \index[authors]{Ballini, Rosangela} \index[authors]{Gomide, Fernando}

\begin{center}
  {\it Leandro Maciel, Rosangela Ballini and Fernando Gomide}
\end{center}
%\begin{minipage}{1\textwidth}
%\end{minipage}

\vskip 0.8cm

 Financial interval time series (ITS) is a time series whose value at each time step is an interval composed by the low and the high price of an asset. The low-high price range is related to the concept of volatility because it inherits intraday price variability. Accurate forecasting of price ranges is essential for derivative pricing, trading strategies, risk management, and portfolio allocation. This paper suggests a fuzzy rule-based approach to model and to forecast interval-valued time series. The model is a collection of functional fuzzy rules with affine consequents capable to express the nonlinear relationships encountered in interval-valued data. An application concerning one-step-ahead forecast of interval-valued EUR/USD exchange rate using actual data is also addressed.  The forecast performance of the fuzzy rule-based model is compared to that of traditional econometric time series methods and alternative interval models employing statistical criteria for both, low and high exchange rate prices. The results show that fuzzy rule-based modeling approach developed in this paper outperforms the random walk, and other competitive approaches in out-of-sample interval-valued exchange rate forecasting.

\keywords{Interval-valued data; exchange rate forecast; fuzzy modeling}



\end{document}
