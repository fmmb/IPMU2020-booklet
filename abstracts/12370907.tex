\documentclass[../booklet.tex]{subfiles}

\begin{document}

\section[Archimedean Choice Functions: an Axiomatic Foundation for Imprecise Decision Making. {\it Jasper De Bock}]{Archimedean Choice Functions: an Axiomatic Foundation for Imprecise Decision Making}
\index[authors]{Bock, Jasper De}

\begin{center}
  {\it Jasper De Bock}
\end{center}
%\begin{minipage}{1\textwidth}
%\end{minipage}

\vskip 0.8cm


%The abstract should briefly summarize the contents of the paper in
%150--250 words.
If uncertainty is modelled by a probability measure, decisions are typically made by choosing the option with the highest expected utility. If an imprecise probability model is used instead, this decision rule can be generalised in several ways. We here focus on two such generalisations that apply to sets of probability measures: E-admissibility and maximality. Both of them can be regarded as special instances of so-called choice functions, a very general mathematical framework for decision making. For each of these two decision rules, we provide a set of necessary and sufficient conditions on choice functions that uniquely characterises this rule, thereby providing an axiomatic foundation for imprecise decision making with sets of probabilities. A representation theorem for Archimedean choice functions in terms of coherent lower previsions lies at the basis of both results.

\keywords{E-admissibility \and Maximality \and Archimedean choice functions \and Decision Making \and Imprecise probabilities.}



\end{document}
