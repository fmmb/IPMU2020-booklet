\documentclass[../booklet.tex]{subfiles}

\begin{document}

\subsection[Measuring polarization: A fuzzy set theoretical approach. {\it Juan Antonio Guevara, Daniel Gómez, José Manuel Robles and Francisco Javier Montero}]{Measuring polarization: A fuzzy set theoretical approach}
\index[authors]{Guevara, Juan Antonio} \index[authors]{Gómez, Daniel} \index[authors]{Robles, José Manuel} \index[authors]{Montero, Francisco Javier}

\begin{center}
  {\it Juan Antonio Guevara, Daniel Gómez, José Manuel Robles and Francisco Javier Montero}
\end{center}
%\begin{minipage}{1\textwidth}
%\end{minipage}

\vskip 0.8cm


The measurement of polarization has been studied over the last thirty years. Despite the different applied approaches, since polarization concept is complex, we find a lack of consensus about how it should be measured. This paper proposes a new approach to the measurement of the polarization phenomenon based on fuzzy set. Fuzzy approach provides a new perspective whose elements admit degrees of membership. Since reality is not black and white, a polarization measure should include this key characteristic. For this purpose we analyze polarization metric properties and develop a new risk of polarization measure using aggregation operators and overlapping functions. We simulate a sample of $N = 391315$ cases across a 5-likert-scale with different distributions to test our measure. Other polarization measures were applied to compare situations where fuzzy set approach offers different results, where membership functions have proved to play an essential role in the measurement. Finally, we want to highlight the new and potential contribution of fuzzy set approach to the polarization measurement which opens a new field to research on.
\keywords{Polarization  \and Fuzzy set \and Ordinal variation.}



\end{document}
