\documentclass[../booklet.tex]{subfiles}

\begin{document}

\subsection[To BERT or not to BERT Dealing with possible BERT failures in an Entailment task. {\it Pedro Fialho, Luísa Coheur and Paulo Quaresma}]{To BERT or not to BERT Dealing with possible BERT failures in an Entailment task}
\index[authors]{Fialho, Pedro} \index[authors]{Coheur, Luísa} \index[authors]{Quaresma, Paulo}

\begin{center}
  {\it Pedro Fialho, Luísa Coheur and Paulo Quaresma}
\end{center}
%\begin{minipage}{1\textwidth}
%\end{minipage}

\vskip 0.8cm


In this paper we focus on an Natural Language Inference task. Being given two sentences, we classify their relation as NEUTRAL, ENTAILMENT or CONTRADICTION. Considering the achievements of BERT (Bidirectional Encoder Representations from Transformers) in many Natural Language Processing tasks, we use BERT features to create our base model for this task. However, several questions arise: can other features improve the performance obtained with BERT? If we are able to predict the situations in which BERT will fail, can we improve the performance by providing alternative models for these situations? We test several strategies and models, as alternatives to the standalone BERT model in the possible failure situations, and we take advantage of semantic features extracted from Discourse Representation Structures.

\keywords{Natural Language Inference \and Feature Engineering \and Failure Prediction Model.}



\end{document}
