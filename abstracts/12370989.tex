\documentclass[../booklet.tex]{subfiles}

\begin{document}

\subsection[Belief Functions for the Importance Assessment in Multiplex Networks. {\it Natalia Meshcheryakova and Alexander Lepskiy}]{Belief Functions for the Importance Assessment in Multiplex Networks}
\index[authors]{Meshcheryakova, Natalia} \index[authors]{Lepskiy, Alexander}

\begin{center}
  {\it Natalia Meshcheryakova and Alexander Lepskiy}
\end{center}
%\begin{minipage}{1\textwidth}
%\end{minipage}

\vskip 0.8cm


We apply Dempster-Shafer theory in order to reveal important elements in undirected weighted networks. We estimate cooperation of each node with different groups of vertices that surround it via construction of belief functions. The obtained intensities of cooperation are further redistributed over all elements of a particular group of nodes that results in pignistic probabilities of node-to-node interactions. Finally, pairwise interactions can be aggregated into the centrality vector that ranks nodes with respect to derived values. We also adapt the proposed model to multiplex networks. In this type of networks nodes can be differently connected with each other on several levels of interaction. Various combination rules help to analyze such systems as a single entity, that has many advantages in the study of complex systems. In particular, Dempster rule takes into account the inconsistency in initial data that has an impact on the final centrality ranking. We also provide a numerical example that illustrates the distinctive features of the proposed model. Additionally, we establish analytical relations between a proposed measure and classical centrality measures for particular graph configurations.

\keywords{Belief Functions  \and Network Analysis \and Centrality Measures.}



\end{document}
