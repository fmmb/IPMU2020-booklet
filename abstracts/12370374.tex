\documentclass[../booklet.tex]{subfiles}

\begin{document}

\subsection[A Graph Theory Approach to Fuzzy Rule Base Simplification. {\it Caro Fuchs, Simone Spolaor, Marco S. Nobile and Uzay Kaymak}]{A Graph Theory Approach to Fuzzy Rule Base Simplification}
\index[authors]{Fuchs, Caro} \index[authors]{Spolaor, Simone} \index[authors]{Nobile, Marco S.} \index[authors]{Kaymak, Uzay}

\begin{center}
  {\it Caro Fuchs, Simone Spolaor, Marco S. Nobile and Uzay Kaymak}
\end{center}
%\begin{minipage}{1\textwidth}
%\end{minipage}

\vskip 0.8cm


Fuzzy inference systems (FIS) gained popularity and found application in several fields of science over the last years, because they are more transparent and interpretable than other common (black-box) machine learning approaches.
However, transparency is not automatically achieved when FIS are estimated from data, thus researchers are actively investigating methods to design interpretable FIS.
Following this line of research, we propose a new approach for FIS simplification which leverages graph theory to identify and remove similar fuzzy sets from rule bases.
We test our methodology on two data sets to show how this approach can be used to simplify the rule base without sacrificing accuracy.

\keywords{Fuzzy Logic \and Takagi--Sugeno Fuzzy Model \and Data-Driven Modeling\and Open-Source Software \and Python \and Graph Theory.}




\end{document}
