\documentclass[../booklet.tex]{subfiles}

\begin{document}

\subsection[Transparency of classification systems for clinical decision support. {\it Antoine Richard, Brice Mayag, François Talbot, Alexis Tsoukias and Yves Meinard}]{Transparency of classification systems for clinical decision support}
\index[authors]{Richard, Antoine} \index[authors]{Mayag, Brice} \index[authors]{Talbot, François} \index[authors]{Tsoukias, Alexis} \index[authors]{Meinard, Yves}

\begin{center}
  {\it Antoine Richard, Brice Mayag, François Talbot, Alexis Tsoukias and Yves Meinard}
\end{center}
%\begin{minipage}{1\textwidth}
%\end{minipage}

%\vskip 0.8cm


  In collaboration with the Civil Hospitals of Lyon, we aim
to develop a "transparent" classification system for medical purposes.
To do so, we need clear definitions and operational criteria to determine
what is a "transparent" classification system in our context.
However, the term "transparency" is often left undefined
in the literature, and there is a lack of operational criteria
allowing to check whether a given algorithm deserves to be called
"transparent" or not.
Therefore, in this paper, we propose a definition of "transparency"
for classification systems in medical contexts.
We also propose several operational criteria to evaluate
whether a classification system can be considered 
"transparent".
We apply these operational criteria to evaluate the 
"transparency" of several well-known classification systems.

\keywords{Explainable AI, Transparency of Algorithms, Health Information
  Systems, Multi-label Classification}



\end{document}
