\documentclass[../booklet.tex]{subfiles}

\begin{document}

\section[A Bidirectional Subsethood Based Fuzzy Measure for Aggregation of Interval-Valued Data. {\it Shaily Kabir and Christian Wagner}]{A Bidirectional Subsethood Based Fuzzy Measure for Aggregation of Interval-Valued Data}
\index[authors]{Kabir, Shaily} \index[authors]{Wagner, Christian}

\begin{center}
  {\it Shaily Kabir and Christian Wagner}
\end{center}
%\begin{minipage}{1\textwidth}
%\end{minipage}

\vskip 0.8cm


Recent advances in the literature have leveraged the fuzzy integral (FI), a powerful multi-source aggregation operator,  where a fuzzy measure (FM) is used to capture the worth of all combinations of subsets of sources. While in most applications, the FM is defined either by experts or numerically derived through optimization, these approaches are only viable if additional information on the sources is available. When such information is unavailable, as is commonly the case when sources are unknown a priori (e.g., in crowdsourcing), prior work has proposed the extraction of valuable insight (captured within FMs) directly from the evidence or input data by analyzing properties such as specificity or agreement amongst sources. Here, existing agreement-based FMs use established measures of similarity such as Jaccard and Dice to estimate the source agreement. Recently, a new similarity measure based on bidirectional subsethood was put forward to compare evidence, minimizing limitations such as \emph{aliasing} (where different inputs result in the same similarity output) present in traditional similarity measures. In this paper, we build on this new similarity measure to develop a new instance of the agreement-based FM for interval-valued data. The proposed FM is purposely designed to support aggregation, and unlike previous agreement FMs, it degrades gracefully to an average operator for cases where no overlap between sources exists. We validate that it respects all requirements of a FM and explore its impact when used in conjunction with the Choquet FI for data fusion as part of both synthetic and real-world datasets, showing empirically that it generates robust and qualitatively superior outputs for the cases considered.

\keywords{Data aggregation \and Fuzzy measures \and Fuzzy integrals \and Subsethood \and Similarity measure \and Interval-valued data.}



\end{document}
