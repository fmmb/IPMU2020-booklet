\documentclass[../booklet.tex]{subfiles}

\begin{document}

\section[Keynote Talk: From Eliza to Siri and beyond. {\it Luísa Coheur}]{Keynote Talk: From Eliza to Siri and beyond}
\index[authors]{Coheur, Luísa}

\begin{center}
  {\it Luísa Coheur}
\end{center}
%\begin{minipage}{1\textwidth}
%\end{minipage}

\vskip 0.8cm

Since Eliza, the first chatbot ever, developed in the 60s, researchers
try to make machines understand (or mimic the understanding) of
Natural Language input. Some conversational agents target small talk,
while others are more task-oriented. However, from the earliest
rule-based systems to the recent data-driven approaches, although many
paths were explored with more or less success, we are not there
yet. Rule-based systems require much manual work; data-driven systems
require a lot of data. Domain adaptation is (again) a current
hot-topic. The possibility to add emotions to the conversational
agents' responses, or to make their answers capture their ``persona'',
are some popular research topics.  This paper explains why the task of
Natural Language Understanding is so complicated, detailing the
linguistic phenomena that lead to the main challenges. Then, the long
walk in this field is surveyed, from the earlier systems to the
current trends.


\end{document}
