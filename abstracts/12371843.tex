\documentclass[../booklet.tex]{subfiles}

\begin{document}

\subsection[Statistical methods for processing neuroimaging data from two different sites with a Down syndrome population application. {\it Davneet Minhas, Zixi Yang, John Muschelli, Charles Laymon, Joseph Mettenburg, Matthew Zammit, Sterling Johnson, Chester Mathis, Ann Cohen, Benjamin Handen, William Klunk, Ciprian Crainiceanu, Bradley Christian and Dana Tudorascu}]{Statistical methods for processing neuroimaging data from two different sites with a Down syndrome population application}
\index[authors]{Minhas, Davneet} \index[authors]{Yang, Zixi} \index[authors]{Muschelli, John} \index[authors]{Laymon, Charles} \index[authors]{Mettenburg, Joseph} \index[authors]{Zammit, Matthew} \index[authors]{Johnson, Sterling} \index[authors]{Mathis, Chester} \index[authors]{Cohen, Ann} \index[authors]{Handen, Benjamin} \index[authors]{Klunk, William} \index[authors]{Crainiceanu, Ciprian} \index[authors]{Christian, Bradley} \index[authors]{Tudorascu, Dana}

\begin{center}
  {\it Davneet Minhas, Zixi Yang, John Muschelli, Charles Laymon, Joseph Mettenburg, Matthew Zammit, Sterling Johnson, Chester Mathis, Ann Cohen, Benjamin Handen, William Klunk, Ciprian Crainiceanu, Bradley Christian and Dana Tudorascu}
\end{center}
%\begin{minipage}{1\textwidth}
%\end{minipage}

%\vskip 0.8cm

Harmonization of magnetic resonance imaging (MRI) and positron emission tomography (PET) scans from multi-scanner and multi-site studies presents a challenging problem. We applied the Removal of Artificial Voxel Effect by Linear regression (RAVEL) method to normalize T1-MRI intensities collected on two different scanners across two different sites as part of the Neurodegeneration in Aging Down syndrome (NiAD) study. The effects on FreeSurfer regional cortical thickness and volume outcome measures, in addition to FreeSurfer-based regional quantification of amyloid PET standardized uptake value ratio (SUVR) outcomes, were evaluated. A neuroradiologist visually assessed the accuracy of FreeSurfer hippocampus segmentations with and without the application of RAVEL. Quantitative results demonstrated that the application of RAVEL intensity normalization prior to running FreeSurfer significantly impacted both FreeSurfer volume and cortical thickness outcome measures. Visual assessment demonstrated that the application of RAVEL significantly improved FreeSurfer hippocampal segmentation accuracy. The RAVEL intensity normalization had little impact on PET SUVR measures.

\keywords{Harmonization, MRI, PET}




\end{document}
