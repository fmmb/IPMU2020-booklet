\documentclass[../booklet.tex]{subfiles}

\begin{document}

\subsection[Combining absolute and relative information with frequency distributions for ordinal classification. {\it Mengzi Tang, Raúl Pérez-Fernández and Bernard De Baets}]{Combining absolute and relative information with frequency distributions for ordinal classification}
\index[authors]{Tang, Mengzi} \index[authors]{Pérez-Fernández,  Raúl} \index[authors]{Baets, Bernard De}

\begin{center}
  {\it Mengzi Tang, Raúl Pérez-Fernández and Bernard De Baets}
\end{center}
%\begin{minipage}{1\textwidth}
%\end{minipage}

\vskip 0.8cm


A large amount of labelled data (absolute information) is usually needed for an ordinal classifier to attain a good performance. As shown in a recent paper by the present authors, the lack of a large amount of absolute information can be overcome by additionally considering some side information in the form of relative information, thus augmenting the method of nearest neighbors. In this paper, we adapt the method of nearest neighbors for dealing with a specific type of relative information: frequency distributions of pairwise comparisons (rather than a single pairwise comparison). We test the proposed method on some classical machine learning datasets and demonstrate its effectiveness.
%The abstract should briefly summarize the contents of the paper in 150--250 words.

\keywords{Ordinal classification \and Nearest neighbors \and Absolute information \and Relative information \and Frequency distributions.}




\end{document}
