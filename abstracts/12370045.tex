\documentclass[../booklet.tex]{subfiles}

\begin{document}

\subsection[Keynote Talk: Average Jane, Where Art Thou? - Recent Avenues in Efficient Machine Learning under Subjectivity Uncertainty. {\it Björn Schuller}]{Keynote Talk: Average Jane, Where Art Thou? - Recent Avenues in Efficient Machine Learning under Subjectivity Uncertainty}
\index[authors]{Schuller, Björn}

\begin{center}
  {\it Björn Schuller}
\end{center}
%\begin{minipage}{1\textwidth}
%\end{minipage}

\vskip 0.8cm


In machine learning tasks an actual `ground truth' may not be available. Then, machines often have to rely on human labelling of data. This becomes challenging the more subjective the learning task is, as human agreement can be low. To cope with the resulting high uncertainty, one could train individual models reflecting a single human's opinion. However, this is not viable, if one aims at mirroring the general opinion of a hypothetical `completely average person' -- the `average Jane'. Here, I summarise approaches to optimally learn efficiently in such a case. First, different strategies of reaching a single learning target from several labellers will be discussed. This includes varying labeller trustability and the case of time-continuous labels with potential dynamics. As human labelling is a labour-intensive endeavour, active and cooperative learning strategies can help reduce the number of labels needed. Next, sample informativeness can be exploited in teacher-based algorithms to additionally weigh data by certainty. In addition, multi-target learning of different labeller tracks in parallel and/or of the uncertainty can help improve the model robustness and provide an additional uncertainty measure. Cross-modal strategies to reduce uncertainty offer another view. From these and further recent strategies, I distil a number of future avenues to handle subjective uncertainty in machine learning. These comprise bigger, yet weakly labelled data processing basing amongst other on reinforcement learning, lifelong learning, and self-learning. Illustrative examples stem from the fields of Affective Computing and Digital Health -- both notoriously marked by subjectivity uncertainty.

% \keywords{Machine Learning  \and Uncertainty \and Subjectivity \and Active Learning \and Cooperative Learning.}
% 


\end{document}
