\documentclass[../booklet.tex]{subfiles}

\begin{document}

\subsection[On the Analysis of Illicit Supply Networks using Variable State Resolution-Markov Chains. {\it Jorge González Ordiano, Lisa Finn, Anthony Winterlich, Gary Moloney and Steven Simske}]{On the Analysis of Illicit Supply Networks using Variable State Resolution-Markov Chains}
\index[authors]{Ordiano, Jorge González} \index[authors]{Finn, Lisa} \index[authors]{Winterlich, Anthony} \index[authors]{Moloney, Gary} \index[authors]{Simske, Steven}

\begin{center}
  {\it Jorge González Ordiano, Lisa Finn, Anthony Winterlich, Gary Moloney and Steven Simske}
\end{center}
%\begin{minipage}{1\textwidth}
%\end{minipage}

\vskip 0.8cm


The trade in illicit items, such as counterfeits, not only leads to the loss of large sums of private and public revenue, but also poses a danger to individuals, undermines governments, and---in the most extreme cases---finances criminal organizations. It is estimated that in 2013 trade in illicit items accounted for $2.5\%$ of the global commerce. To combat illicit trade, it is necessary to understand its illicit supply networks. Therefore, we present in this article an approach that is able to find an optimal description of an illicit supply network using a series of Variable State Resolution-Markov Chains. The new method is applied to a real-world dataset stemming from the Global Product Authentication Service of Micro Focus International. The results show how an illicit supply network might be analyzed with the help of this method.

\keywords{Data Mining \and Markov Chain \and Illicit Trade}



\end{document}
