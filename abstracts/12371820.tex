\documentclass[../booklet.tex]{subfiles}

\begin{document}

\subsection[High Dimensional Bayesian Regularization in Regressions Involving Symmetric Tensors. {\it Rajarshi Guhaniyogi}]{High Dimensional Bayesian Regularization in Regressions Involving Symmetric Tensors}
\index[authors]{Guhaniyogi, Rajarshi}

\begin{center}
  {\it Rajarshi Guhaniyogi}
\end{center}
%\begin{minipage}{1\textwidth}
%\end{minipage}

%\vskip 0.8cm


This article develops a regression framework with a symmetric tensor response and vector predictors. The existing literature involving symmetric tensor response and vector predictors proceeds by vectorizing the tensor response to a multivariate vector, thus ignoring the structural information in the tensor. A few recent approaches have proposed novel regression frameworks exploiting the structure of the symmetric tensor and assume symmetric tensor coefficients corresponding to scalar predictors to be low-rank. Although low-rank constraint on coefficient tensors are computationally efficient, they might appear to be restrictive in some real data applications.  Motivated by this, we propose a novel class of regularization or shrinkage priors for the symmetric tensor coefficients. Our modeling framework a-priori expresses a symmetric tensor coefficient as sum of low rank and sparse structures, with both these structures being suitably regularized using Bayesian regularization techniques. The proposed framework allows identification of tensor nodes significantly influenced by each scalar predictor. Our framework is implemented using an efficient Markov Chain Monte Carlo algorithm. Empirical results in simulation studies show competitive performance of the proposed approach over its competitors.

\keywords{Low-rank Structure, Symmetric Tensor Predictor, Shrinkage Prior, Spike and Slab Prior.}



\end{document}
