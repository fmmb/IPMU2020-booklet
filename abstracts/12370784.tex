\documentclass[../booklet.tex]{subfiles}

\begin{document}

\subsection[On the elicitation of an optimal outer approximation of a coherent lower probability. {\it Enrique Miranda, Ignacio Montes and Paolo Vicig}]{On the elicitation of an optimal outer approximation of a coherent lower probability}
\index[authors]{Miranda, Enrique} \index[authors]{Montes, Ignacio} \index[authors]{Vicig, Paolo}

\begin{center}
  {\it Enrique Miranda, Ignacio Montes and Paolo Vicig}
\end{center}
%\begin{minipage}{1\textwidth}
%\end{minipage}

%\vskip 0.8cm


The process of outer approximating a coherent lower probability by a more tractable model with additional properties, such as 2- or completely monotone capacities, may not have a unique solution. In this paper, we investigate whether a number of approaches may help in eliciting a unique outer approximation: minimising a number of distances with respect to the initial model, or maximising the specificity of the outer approximation. We apply these to 2- and completely monotone approximating lower probabilities, and also to possibility measures.
%The abstract should briefly summarize the contents of the paper in 150--250 words.

\keywords{Coherent lower probabilities, 2-monotonicity, Belief functions, Possibility measures, Specificity.}



\end{document}
