\documentclass[../booklet.tex]{subfiles}

\begin{document}

\section[Towards multi-perspective conformance checking with aggregation operations. {\it Sicui Zhang, Laura Genga, Lukas Dekker, Hongchao Nie, Xudong Lu, Huilong Duan and Uzay Kaymak}]{Towards multi-perspective conformance checking with aggregation operations}
\index[authors]{Zhang, Sicui} \index[authors]{Genga,  Laura} \index[authors]{Dekker,  Lukas} \index[authors]{Nie,  Hongchao} \index[authors]{Lu,  Xudong} \index[authors]{Duan,  Huilong} \index[authors]{Kaymak, Uzay}

\begin{center}
  {\it Sicui Zhang, Laura Genga, Lukas Dekker, Hongchao Nie, Xudong Lu, Huilong Duan and Uzay Kaymak}
\end{center}
%\begin{minipage}{1\textwidth}
%\end{minipage}

\vskip 0.8cm


 Conformance checking techniques are widely adopted to validate process executions against a set of constraints describing the expected behavior. However, most approaches adopt a crisp evaluation of
 deviations, with the result that small violations are considered at the same level of significant ones. %This affects the quality of the provided diagnostics, especially when there exists some tolerance with respect to reasonably small violations, and hampers the flexibility of the process.
 Furthermore, in the presence of multiple data constraints the overall deviation severity is assessed by summing up each single deviation. This approach easily leads to misleading diagnostics; furthermore, it does not take into account user's needs, that are likely to differ depending on the context of the analysis.
%Previous work has proposed a novel approach which allows to represent actors' tolerance with respect to violations and to account for severity of deviations when assessing executions compliance. However, it applies the basic strategy of standard conformance checking techniques for dealing with multiple constraints deviations. To address this issue, 
We propose a novel methodology based on the use of aggregation functions, to assess the level of deviation severity for a set of constraints, and to customize the tolerance to deviations of multiple constraints.
 %
\keywords{conformance checking \and fuzzy aggregation \and data perspective }



\end{document}
