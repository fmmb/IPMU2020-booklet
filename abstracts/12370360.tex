\documentclass[../booklet.tex]{subfiles}

\begin{document}

\subsection[Using Topic Information to Improve Non-Exact Keyword-Based Search for Mobile Applications. {\it Eugénio Ribeiro, Ricardo Ribeiro, Fernando Batista and João Oliveira}]{Using Topic Information to Improve Non-Exact Keyword-Based Search for Mobile Applications}
\index[authors]{Ribeiro, Eugénio} \index[authors]{Ribeiro,  Ricardo} \index[authors]{Batista,  Fernando} \index[authors]{Oliveira, João}

\begin{center}
  {\it Eugénio Ribeiro, Ricardo Ribeiro, Fernando Batista and João Oliveira}
\end{center}
%\begin{minipage}{1\textwidth}
%\end{minipage}

\vskip 0.8cm


Considering the wide offer of mobile applications available nowadays, effective search engines are imperative for an user to find applications that provide a specific desired functionality. Retrieval approaches that leverage topic similarity between queries and applications have shown promising results in previous studies. However, the search engines used by most app stores are based on keyword-matching and boosting. In this paper, we explore means to include topic information in such approaches, in order to improve their ability to retrieve relevant applications for non-exact queries, without impairing their computational performance. More specifically, we create topic models specialized on application descriptions and explore how the most relevant terms for each topic covered by an application can be used to complement the information provided by its description. Our experiments show that, although these topic keywords are not able to provide all the information of the topic model, they provide a sufficiently informative summary of the topics covered by the descriptions, leading to improved performance.

\keywords{Application search \and Topic information \and Non-exact queries.}


\end{document}
