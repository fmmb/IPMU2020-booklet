\documentclass[../booklet.tex]{subfiles}

\begin{document}

\section[A Belief Classification Approach based on Artificial Immune Recognition System. {\it Rihab Abdelkhalek and Zied Elouedi}]{A Belief Classification Approach based on Artificial Immune Recognition System}
\index[authors]{Abdelkhalek, Rihab} \index[authors]{Elouedi, Zied}

\begin{center}
  {\it Rihab Abdelkhalek and Zied Elouedi}
\end{center}
%\begin{minipage}{1\textwidth}
%\end{minipage}

\vskip 0.8cm








Artificial Immune Recognition Systems (AIRS) are  supervised
classification methods inspired by the immune system metaphors. They enjoy a great popularity in the filed of machine learning by achieving good and competitive classification results. Nonetheless, while these approaches work properly under a certain framework, they
present some weaknesses basically related to their inability to deal with uncertainty. This is considered as an important challenge in real-world classification problems. Furthermore, using traditional AIRS approaches, all memory cells  are considered with the same importance during the classification process which may affect the final generated results. To tackle these issues, we propose in this paper a new AIRS approach under the belief function framework. Our approach tends to handle the uncertainty pervaded in the classification stage while taking into account the number of training antigens represented by each memory
cell. The performance of the proposed
evidential AIRS approach is validated on real-world data sets and compared to state of the art AIRS under certain and uncertain frameworks.



\keywords{Machine learning \and Classification  \and · Artificial immune recognition systems \and Uncertainty \and Belief function theory.}



\end{document}
