\documentclass[../booklet.tex]{subfiles}

\begin{document}

\subsection[Image-based World-perceiving Knowledge Graph (WpKG) with Imprecision. {\it Navid Rezaei, Marek Z. Reformat and Ronald R. Yager}]{Image-based World-perceiving Knowledge Graph (WpKG) with Imprecision}
\index[authors]{Rezaei, Navid} \index[authors]{Reformat,  Marek Z.} \index[authors]{Yager, Ronald R.}

\begin{center}
  {\it Navid Rezaei, Marek Z. Reformat and Ronald R. Yager}
\end{center}
%\begin{minipage}{1\textwidth}
%\end{minipage}

\vskip 0.8cm


%This work introduces a systematic approach to automatically generate a probabilistic knowledge graph from only visual data. The process uses trained neural networks predict entities and their relations for each image. The data is then aggregated to a probabilistic knowledge graph. The entities and relations can also be disambiguated using WordNet data. We have shown qualitatively that this knowledge graph can extract situational common sense knowledge for everyday objects through first-order and second-order relations.

Knowledge graphs are a data format that enables the representation of semantics. Most of the available graphs focus on the representation of facts, their features, and relations between them. However, from the point of view of possible applications of semantically rich data formats in intelligent, real-world scenarios, there is a need for knowledge graphs that describe contextual information regarding realistic and casual relations between items in the real world.

In this paper, we present a methodology of generating knowledge graphs addressing such a need. We call them {\em World-perceiving} {\em Knowledge} {\em Gra\-phs} -- {\em WpKG}. The process of their construction is based on analyzing images. We apply deep learning image processing methods to extract scene graphs. We  combine these graphs, and process the obtained graph to determine  importance of relations between items detected on the images. The generated WpKG is used as a basis for constructing possibility graphs. We illustrate the process and show some snippets of the generated knowledge and possibility graphs.


\keywords{Knowledge Graph \and Deep Learning \and Common Sense \and Possibility Theory.}



\end{document}
