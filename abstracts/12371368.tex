\documentclass[../booklet.tex]{subfiles}

\begin{document}

\subsection[A Fuzzy Goal Programming Approach To Fully Fuzzy Linear Regression. {\it Boris Pérez-Cañedo, Alejandro Rosete, José Luis Verdegay and Eduardo René Concepción-Morales}]{A Fuzzy Goal Programming Approach To Fully Fuzzy Linear Regression}
\index[authors]{Pérez-Cañedo, Boris} \index[authors]{Rosete, Alejandro} \index[authors]{Verdegay, José Luis} \index[authors]{Concepción-Morales, Eduardo René}

\begin{center}
  {\it Boris Pérez-Cañedo, Alejandro Rosete, José Luis Verdegay and Eduardo René Concepción-Morales}
\end{center}
%\begin{minipage}{1\textwidth}
%\end{minipage}

%\vskip 0.8cm


Traditional linear regression analysis aims at finding a linear functional relationship between predictor and response variables based on available data of a given system, and, when this relationship is found, it is used to predict the future behaviour of the system. The difference between the observed and predicted data is supposed to be due to measurement errors. In fuzzy linear regression, on the other hand, this difference is supposed to be mainly due to the indefiniteness of the system. In this paper, we assume that predictor and response variables are LR-type fuzzy numbers, and so are all regression coefficients; this is known as fully fuzzy linear regression (FFLR) problem. We transform the FFLR problem into a fully fuzzy multiobjective linear programming (FFMOLP) problem. Two fuzzy goal programming methods based on linear and Chebyshev scalarisations are proposed to solve the FFMOLP problem. The proposed methods are compared with a recently published method and show promising results.

\keywords{fully fuzzy linear regression, fully fuzzy multiobjective linear programming, fuzzy goal programming, linear scalarisation, Chebyshev scalarisation.}



\end{document}
