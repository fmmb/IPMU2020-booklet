\documentclass[../booklet.tex]{subfiles}

\begin{document}

\section[A Word Embedding Model for Mapping Food Composition Databases using Fuzzy Logic. {\it Andrea Morales-Garzón, Juan Gomez-Romero and Maria J. Martin-Bautista}]{A Word Embedding Model for Mapping Food Composition Databases using Fuzzy Logic}
\index[authors]{Morales-Garzón, Andrea} \index[authors]{Gomez-Romero,  Juan} \index[authors]{Martin-Bautista, Maria J.}

\begin{center}
  {\it Andrea Morales-Garzón, Juan Gomez-Romero and Maria J. Martin-Bautista}
\end{center}
%\begin{minipage}{1\textwidth}
%\end{minipage}

\vskip 0.8cm


%The abstract should briefly summarize the contents of the paper in
%\\150--250 words.

% --------------------------ANTIGUA VERSIÓN DEL ABSTRACT:--------------------------------------
%Nutrition data are fundamental to support AI-based Food Computing, ranging from diet recommendation systems to understanding the culinary culture. A primary resource is the Nutrient Database of the United States Department of Agriculture (USDA), an open repository that provides composition data of a wide variety of foods and food products. The USDA database is considered the gold standard for nutrition data and the reference for several other public and private resources. This is the case of i-Diet, a proprietary information system used within the H2020 European project Stance4Health to support personalized nutrition considering the specifics of the Spanish and other European regional cuisines. This paper addresses the problem of linking equivalent items between two databases based on their textual attributes. Specifically, we will apply this technique to match the elements of the two previously mentioned food composition databases by calculating the most likely match of each item in another given database. A number of experiments have been carried by employing different kind of metrics, some of  them  involving  Fuzzy  Logic.  The  experiments  show  that  the  mappings are highly accurate, fuzzy logic highly improve the robustness of the  model  and  most  errors  are  due  to  incorrect  descriptions  of  i-Diet Spanish items in English.


This paper addresses the problem of mapping equivalent items between two databases based on their textual descriptions. Specifically, we will apply this technique to link the elements of two food composition databases by calculating the most likely match of each item in another given database. A number of experiments have been carried by employing different distance metrics, some of them involving Fuzzy Logic. The experiments show that the mappings are highly accurate and Fuzzy Logic improves the precision of the model.% and most errors are due to incorrect descriptions of i-Diet Spanish items in English.


\keywords{Word embedding  \and Fuzzy distance \and Database Alignment}



\end{document}
