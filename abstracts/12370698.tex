\documentclass[../booklet.tex]{subfiles}

\begin{document}

\section[Feature Extraction with TF-IDF and Game-theoretic Shadowed Sets. {\it Yan Zhang, Yue Zhou and Jingtao Yao}]{Feature Extraction with TF-IDF and Game-theoretic Shadowed Sets}
\index[authors]{Zhang, Yan} \index[authors]{Zhou,  Yue} \index[authors]{Yao, Jingtao}

\begin{center}
  {\it Yan Zhang, Yue Zhou and Jingtao Yao}
\end{center}
%\begin{minipage}{1\textwidth}
%\end{minipage}

\vskip 0.8cm



TF-IDF is one of the most commonly used weighting metrics for measuring the relationship of words to documents. It is widely used for word feature extraction. In many research and applications, the thresholds of TF-IDF for selecting relevant words are only based on trial or experiences. Some cut-off strategies have been proposed in which the thresholds are selected based on Zipf's law or feedbacks from model performances.
However, the existing approaches are restricted in specific domains or tasks, and they ignore the imbalance of the number of representative words in different categories of documents. To address these issues, we apply game-theoretic shadowed set model to select the word features given TF-IDF information.
Game-theoretic shadowed sets determine the thresholds of TF-IDF using game theory and repetition learning mechanism.
Experimental results on real world new category dataset show that our model not only outperforms all baseline cut-off approaches, but also speed up the classification algorithms.


\keywords{Feature Extraction\and TF-IDF\and Text Classification \and Game-theoretic Shadowed Sets}



\end{document}
