\documentclass[../booklet.tex]{subfiles}

\begin{document}

\subsection[Contextualizing Naive Bayes Predictions. {\it Marcelo Loor and Guy De Tré}]{Contextualizing Naive Bayes Predictions}
\index[authors]{Loor, Marcelo} \index[authors]{Tré, Guy De}

\begin{center}
  {\it Marcelo Loor and Guy De Tré}
\end{center}
%\begin{minipage}{1\textwidth}
%\end{minipage}

\vskip 0.8cm


A classification process can be seen as a set of actions by which several objects are evaluated in order to predict the class(es) those objects belong to. %
%
In situations where transparency is a necessary condition, predictions resulting from a classification process are needed to be interpretable. % 
%
In this paper, we propose a novel variant of a naive Bayes (NB) classification process that yields such interpretable predictions. %
%
In the proposed variant, augmented appraisal degrees (AADs) are used for the contextualization of the evaluations carried out to make the predictions. %
%
Since an AAD has been conceived as a mathematical representation of the connotative meaning in an experience-based evaluation, the incorporation of AADs into a NB classification process helps to put the resulting predictions in context. %
%
An illustrative example, in which the proposed version of NB classification is used for the categorization of newswire articles, shows how such contextualized predictions can favor their interpretability. %
%

\keywords{Explainable artificial intelligence  \and Augmented appraisal degrees \and Naive Bayes classification \and Context handling.}



\end{document}
