\documentclass[../booklet.tex]{subfiles}

\begin{document}

\subsection[A method to generate soft reference data for topic identication. {\it Daniel Vélez, Guillermo Villarino Martínez, J. Tinguaro Rodríguez and Daniel Gomez}]{A method to generate soft reference data for topic identication}
\index[authors]{Vélez, Daniel} \index[authors]{Martínez, Guillermo Villarino} \index[authors]{Rodríguez, J. Tinguaro} \index[authors]{Gomez, Daniel}

\begin{center}
  {\it Daniel Vélez, Guillermo Villarino Martínez, J. Tinguaro Rodríguez and Daniel Gomez}
\end{center}
%\begin{minipage}{1\textwidth}
%\end{minipage}

%\vskip 0.8cm


Text mining and topic identification models are becoming increasingly relevant to extract value from the huge amount of unstructured textual information that companies obtain from their users and clients nowadays. Soft approaches to these problems are also gaining relevance, as in some contexts it may be unrealistic to assume that any document has to be associated to a single topic without any further consideration of the involved uncertainties. However, there is an almost total lack of reference documents allowing a proper assessment of the performance of soft classifiers in such soft topic identification tasks. To address this lack, in this paper a method is proposed that generates topic identification reference documents with a soft but objective nature, and which proceeds by combining, in random but known proportions, phrases of existing documents dealing with different topics. We also provide a computational study illustrating the application of the proposed method on a well-known benchmark for topic identification, as well as showing the possibility of carrying out an informative evaluation of soft classifiers in the context of soft topic identification.

\keywords{Soft classification, Text mining, Topic identification.}



\end{document}
