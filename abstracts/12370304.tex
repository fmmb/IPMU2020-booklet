\documentclass[../booklet.tex]{subfiles}

\begin{document}

\subsection[Forecasting Electricity Consumption in Residential Buildings for home Energy Management Systems. {\it Karol Bot, Antonio Ruano and Maria Da Graça Ruano}]{Forecasting Electricity Consumption in Residential Buildings for home Energy Management Systems}
\index[authors]{Bot, Karol} \index[authors]{Ruano, Antonio} \index[authors]{Ruano, Maria Da Graça}

\begin{center}
  {\it Karol Bot, Antonio Ruano and Maria Da Graça Ruano}
\end{center}
%\begin{minipage}{1\textwidth}
%\end{minipage}

\vskip 0.8cm

Prediction of the energy consumption is a key aspect of home energy man-agement systems, whose aim is to increase the occupant's comfort while re-ducing the energy consumption. This work, employing three years measured data, uses radial basis function neural networks, designed using a multi-objective genetic algorithm (MOGA) framework, for the prediction of total electric power consumption, HVAC demand and other loads demand. The prediction horizon desired is 12 hours, using 15 minutes step ahead model, in a multi-step ahead fashion. To reduce the uncertainty, making use of the preferred set MOGA output, a model ensemble technique is proposed which achieves excellent forecast results, comparing additionally very fa-vorably with existing approaches.
\keywords{Home consumption forecasting, HVAC consumption forecasting, predic-tion methods, neural networks, multi-objective optimization, home energy management systems, ensemble modelling}


\end{document}
