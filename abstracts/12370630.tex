\documentclass[../booklet.tex]{subfiles}

\begin{document}

\subsection[Dynamic pricing using Thompson Sampling with fuzzy events. {\it Jason Rhuggenaath, Paulo Roberto de Oliveira da Costa, Yingqian Zhang, Alp Akcay and Uzay Kaymak}]{Dynamic pricing using Thompson Sampling with fuzzy events}
\index[authors]{Rhuggenaath, Jason} \index[authors]{Costa, Paulo Roberto de Oliveira da} \index[authors]{Zhang, Yingqian} \index[authors]{Akcay, Alp} \index[authors]{Kaymak, Uzay}

\begin{center}
  {\it Jason Rhuggenaath, Paulo Roberto de Oliveira da Costa, Yingqian Zhang, Alp Akcay and Uzay Kaymak}
\end{center}
%\begin{minipage}{1\textwidth}
%\end{minipage}

%\vskip 0.8cm


In this paper we study  a repeated posted-price auction between a single seller and a single buyer that interact for a finite number of periods or rounds. In each round, the seller offers the same item for sale to the buyer. The seller announces a price and the buyer can decide to buy the item at the announced price or the buyer can decide not to buy the item. In this paper we study the problem from the perspective of the buyer who only gets to observe a stochastic measurement of the valuation of the item after he buys the item. Furthermore, in our model the buyer uses fuzzy sets to describe his satisfaction with the observed valuations and he uses fuzzy sets to describe his dissatisfaction with the observed price. In our problem, the buyer makes decisions based on the probability of a fuzzy event. His decision to buy or not depends on whether the satisfaction from having a high enough valuation for the item out weights the dissatisfaction of the quoted price.  We propose an algorithm based on Thompson Sampling and demonstrate that is performs well using numerical experiments. 
\keywords{dynamic pricing, Bayesian modeling, exploration-exploitation trade-off, probability of fuzzy events}



\end{document}
