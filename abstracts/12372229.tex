\documentclass[../booklet.tex]{subfiles}

\begin{document}

\subsection[Fast Convergence of Competitive Spiking Neural Networks with Sample-Based Weight Initialization. {\it Paolo Gabriel Cachi Delgado, Sebastián Ventura and Krzysztof Cios}]{Fast Convergence of Competitive Spiking Neural Networks with Sample-Based Weight Initialization}
\index[authors]{Delgado, Paolo Gabriel Cachi} \index[authors]{Ventura, Sebastián} \index[authors]{Cios, Krzysztof}

\begin{center}
  {\it Paolo Gabriel Cachi Delgado, Sebastián Ventura and Krzysztof Cios}
\end{center}
%\begin{minipage}{1\textwidth}
%\end{minipage}

%\vskip 0.8cm


Recent work on spiking neural networks showed good progress towards unsupervised feature learning. In particular, networks called Competitive Spiking Neural Networks (CSNN) achieve reasonable accuracy in classification tasks. However, two major disadvantages limit their practical applications: high computational complexity and slow convergence. While the first problem has partially been addressed with the development of neuromorphic hardware, no work has addressed the latter problem. In this paper we show that the number of samples the CSNN needs to converge can be reduced significantly by a proposed new weight initialization. The proposed method uses input samples as initial values for the connection weights. Surprisingly, this simple initialization reduces the number of training samples needed for convergence by an order of magnitude without loss of accuracy. We use the MNIST dataset to show that the method is robust even when not all classes are seen during initialization.

\keywords{Spiking Neural Networks, Competitive Learning, Unsupervised Feature Learning.}



\end{document}
