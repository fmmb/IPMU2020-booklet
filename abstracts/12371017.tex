\documentclass[../booklet.tex]{subfiles}

\begin{document}

\subsection[Evaluation of Probabilistic Transformations For Evidential Data Association. {\it Mohammed Boumediene and Jean Dezert}]{Evaluation of Probabilistic Transformations For Evidential Data Association}
\index[authors]{Boumediene, Mohammed} \index[authors]{Dezert, Jean}

\begin{center}
  {\it Mohammed Boumediene and Jean Dezert}
\end{center}
%\begin{minipage}{1\textwidth}
%\end{minipage}

\vskip 0.8cm


Data association is one of the main tasks to achieve in perception applications. Its aim is to match the sensor detections to the known objects. To treat such issue, recent research focus on the evidential approach using belief functions, which are interpreted as an extension of the probabilistic model for reasoning about uncertainty. The data fusion process begins by quantifying sensor data by belief masses. Thereafter, these masses are combined in order to provide more accurate information. Finally, a probabilistic approximation of these combined masses is done to make-decision on associations. Several probabilistic transformations have been proposed in the literature. However, to the best of our knowledge, these transformations have been evaluated only on simulated examples. For this reason, the objective of this paper is to benchmark most of interesting probabilistic transformations on real-data in order to evaluate their performances for the autonomous vehicle perception problematic. 
\keywords{Data Association \and Evidential Theory  \and Belief Functions \and Probabilistic Transformation.}



\end{document}
