\documentclass[../booklet.tex]{subfiles}

\begin{document}

\subsection[Investigation of Ranking Methods within the Military Value of Information (VoI) Problem Domain. {\it Behrooz Etesamipour and Robert J. Hammell II}]{Investigation of Ranking Methods within the Military Value of Information (VoI) Problem Domain}
\index[authors]{Etesamipour, Behrooz} \index[authors]{II, Robert J. Hammell}

\begin{center}
  {\it Behrooz Etesamipour and Robert J. Hammell II}
\end{center}
%\begin{minipage}{1\textwidth}
%\end{minipage}

%\vskip 0.8cm

Determining the relative importance among vast amounts of individual pieces of information is a challenge in the military environment. By aggregating various military intelligence experts' knowledge, decision support tools can be created. A next step in the continuing research in this area is to investigate the use of three prominent ranking methods for aggregating opinions of military intelligence analysts with respect to the Value of Information (VoI) problem domain. This paper offers discussion about ongoing VoI research and demonstrates outcomes from a military-related experiment using Borda count, Condorcet voting, and Instant-runoff voting (IRV) methods as ranking aggregation models. These ranking methods are compared to the "ground truth" as generated by the current fuzzy-based VoI prototype system. The results by incorporating the ranking models on the experiment's data demonstrate the efficacy of these methods in aggregating Subject Matter Expert (SME) opinions and clearly demonstrate the "wisdom of the crowd" effect. Implications related to ongoing VoI research are discussed along with future research plans.
\keywords{value of information, decision support, information aggregation, Borda count, Condorcet voting, Instant-runoff voting, rank aggregation}


\end{document}
