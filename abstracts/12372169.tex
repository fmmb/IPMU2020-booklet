\documentclass[../booklet.tex]{subfiles}

\begin{document}

\subsection[Fuzzy relatinal mathematical morphology: erosion and dilation. {\it Alexander Sostak, Ingrida Uljane and Patrik Eklund}]{Fuzzy relatinal mathematical morphology: erosion and dilation}
\index[authors]{Sostak, Alexander} \index[authors]{Uljane, Ingrida} \index[authors]{Eklund, Patrik}

\begin{center}
  {\it Alexander Sostak, Ingrida Uljane and Patrik Eklund}
\end{center}
%\begin{minipage}{1\textwidth}
%\end{minipage}

\vskip 0.8cm

 In the recent years, the subject if fuzzy mathematical morphology entered the field of interest of many researchers. In our recent paper [REF], we have have developed the basis of the (unstructured) $L$-fuzzy relation mathematical morphology where $L$ is a quantale. In this paper we extend  it to the structured case. We introduce structured $L$-fuzzy relational erosion and dilation operators, study their basic properties, show that under some conditions  these operators are dual and form an adjunction pair. Basing on the topological interpretation of these operators, we introduce the category of  $L$-fuzzy relational morphological spaces and their continuous transformations. 


\keywords{  $L$-fuzzy relational erosion,  $L$-fuzzy relational dilation,  $L$-fuzzy relational morphological spaces, duality, adjointness, continuous transformations}



\end{document}
