\documentclass[../booklet.tex]{subfiles}

\begin{document}

\section[Concept Membership Modeling Using a Choquet Integral. {\it Grégory Smits, Ronald R. Yager, Marie-Jeanne Lesot and Olivier Pivert}]{Concept Membership Modeling Using a Choquet Integral}
\index[authors]{Smits, Grégory} \index[authors]{Yager,  Ronald R.} \index[authors]{Lesot,  Marie-Jeanne} \index[authors]{Pivert, Olivier}

\begin{center}
  {\it Grégory Smits, Ronald R. Yager, Marie-Jeanne Lesot and Olivier Pivert}
\end{center}
%\begin{minipage}{1\textwidth}
%\end{minipage}

\vskip 0.8cm

\newcommand{\nom}{CHOCOLATE}

  Imprecise and subjective concepts, as e.g. {\it promising students}, 
  may be used within data mining tasks or database queries to faithfully describe
  data properties of interest.
  However, defining these concepts is a demanding task for
  the end-user. We thus provide a strategy, called \nom, that
  only requires the user to give a tiny subset of data points that
  are representative of the concept he/she has in mind, and that
  infers a membership function from them. This function may then be
  used to retrieve, from the whole dataset, a ranked list of points that satisfy the concept of interest. \nom\ relies on a
  Choquet integral to aggregate the relevance of individual attribute
  values among all the representative points as well as the
  representativity of sets of such attribute values.  As a
  consequence, a valuable property of the proposed approach is that it
  is able to both capture properties shared by most of the
  user-selected representative data points as well as specific
  properties possessed by only one specific representative data point.
  \keywords{Fuzzy concept, fuzzy measure, Choquet integral}



\end{document}
