\documentclass[../booklet.tex]{subfiles}

\begin{document}

\subsection[Artificial Bee Colony Algorithm Applied toDynamic Flexible Job Shop Problems. {\it Inês Ferreira, Bernardo Firme, Miguel Martins, Tiago Coito, Joaquim Viegas, João Figueiredo, Susana Vieira and João Sousa}]{Artificial Bee Colony Algorithm Applied toDynamic Flexible Job Shop Problems}
\index[authors]{Ferreira, Inês} \index[authors]{Firme,  Bernardo} \index[authors]{Martins,  Miguel} \index[authors]{Coito,  Tiago} \index[authors]{Viegas,  Joaquim} \index[authors]{Figueiredo,  João} \index[authors]{Vieira,  Susana} \index[authors]{Sousa, João}

\begin{center}
  {\it Inês Ferreira, Bernardo Firme, Miguel Martins, Tiago Coito, Joaquim Viegas, João Figueiredo, Susana Vieira and João Sousa}
\end{center}
%\begin{minipage}{1\textwidth}
%\end{minipage}

\vskip 0.8cm


This work introduces a scheduling technique using the Artificial Bee Colony (ABC) algorithm for static and dynamic environments.
The ABC algorithm combines different initial populations and generation of new food source methods, including a moving operations technique and a local search method increasing the variable neighbourhood search that, as a result, improves the solution quality. The algorithm is validated and its performance is tested in a static environment in 9 instances of Flexible Job Shop Problem (FJSP) from Brandimarte dataset obtaining in 5 instances the best known for the instance under study and a new best known in instance mk05.  
The work also focus in developing tools to process the information on the factory through the development of solutions when facing disruptions and dynamic events. Three real-time events are considered on the dynamic environment: jobs cancellation, operations cancellation and new jobs arrival. Two scenarios are studied for each real-time event: the first situation considers the minimization of the disruption between the previous schedule and the new one and the second situation generates a completely new schedule after the occurrence. Summarizing, six adaptations of ABC algorithm are created to solve dynamic environment scenarios and their performances are compared with the benchmarks of two case studies outperforming both. 

\keywords{Dynamic environment \and New Jobs Arrival \and Operations Cancellation \and Jobs Cancellation \and Flexible Job Shop Rescheduling}



\end{document}
