\documentclass[../booklet.tex]{subfiles}

\begin{document}

\subsection[Limit Behaviour of Upper and Lower Expected Time Averages in Discrete-Time Imprecise Markov Chains. {\it Natan T'Joens and Jasper De Bock}]{Limit Behaviour of Upper and Lower Expected Time Averages in Discrete-Time Imprecise Markov Chains}
\index[authors]{T'Joens, Natan} \index[authors]{Bock, Jasper De}

\begin{center}
  {\it Natan T'Joens and Jasper De Bock}
\end{center}
%\begin{minipage}{1\textwidth}
%\end{minipage}

%\vskip 0.8cm


We study the limit behaviour of upper and lower bounds on expected time averages in imprecise Markov chains; a generalised type of Markov chain where the local dynamics, traditionally characterised by transition probabilities, are now represented by sets of `plausible' transition probabilities.
Our main result is a necessary and sufficient condition under which these upper and lower bounds, called upper and lower expected time averages, will converge as time progresses towards infinity to limit values that do not depend on the process' initial state.
Remarkably, our condition is considerably weaker than those needed to establish similar results for so-called limit---or steady state---upper and lower expectations, which are often used to provide approximate information about the limit behaviour of time averages as well.
We show that such an approximation is sub-optimal and that it can be significantly improved by directly using upper and lower expected time averages.

\keywords{Imprecise Markov chain, Upper expectation, Upper transition operator, Expected time average, Weak Ergodicity.}

\end{document}
