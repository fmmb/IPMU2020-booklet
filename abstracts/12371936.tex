\documentclass[../booklet.tex]{subfiles}

\begin{document}

\section[Enhancing the efficiency of the interval-valued fuzzy rule-based classifier with tuning and rule selection. {\it José Antonio Sanz, Tiago da Cruz Asmus, Borja de la Osa and Humberto Bustince}]{Enhancing the efficiency of the interval-valued fuzzy rule-based classifier with tuning and rule selection}
\index[authors]{Sanz, José Antonio} \index[authors]{Asmus,  Tiago da Cruz} \index[authors]{Osa,  Borja de la} \index[authors]{Bustince, Humberto}

\begin{center}
  {\it José Antonio Sanz, Tiago da Cruz Asmus, Borja de la Osa and Humberto Bustince}
\end{center}
%\begin{minipage}{1\textwidth}
%\end{minipage}

\vskip 0.8cm



Interval-Valued fuzzy rule-based classifier with TUning and Rule Selection, IVTURS, is a state-of-the-art fuzzy classifier.  One of the key point of this method is the usage of interval-valued restricted equivalence functions because their parametrization allows one to tune them to each problem, which leads to obtaining accurate results. However, they require the application of the exponentiation several times to obtain a result, which is a time demanding operation implying an extra charge to the computational burden of the method. 

In this contribution, we propose to reduce the number of exponentiation operations executed by the system, so that the efficiency of the method is enhanced with no alteration of the obtained results. Moreover, the new approach also allows for a reduction on the search space of the evolutionary method carried out in IVTURS. Consequently, we also propose four different approaches to take advantage of this reduction on the search space to study if it can imply an enhancement of the accuracy of the classifier. The experimental results prove: 1) the enhancement of the efficiency of IVTURS and 2) the accuracy of IVTURS is competitive versus that of the approaches using the reduced search space.

%are obtained using the same setting than that used in the paper where IVTURS was defined and they

%The abstract should briefly summarize the contents of the paper in 15--250 words.

\keywords{Interval-Valued Fuzzy Rule-based Classification Systems  \and Interval-Valued Fuzzy Sets \and Interval Type-2 Fuzzy Sets \and Evolutionary Fuzzy Systems.}



\end{document}
