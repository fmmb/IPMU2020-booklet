\documentclass[../booklet.tex]{subfiles}

\begin{document}

\subsection[Converting Possibilistic Networks by using Uncertain Gates. {\it Guillaume Petiot}]{Converting Possibilistic Networks by using Uncertain Gates}
\index[authors]{Petiot, Guillaume}

\begin{center}
  {\it Guillaume Petiot}
\end{center}
%\begin{minipage}{1\textwidth}
%\end{minipage}

%\vskip 0.8cm


The purpose of this paper is to define a general frame to convert the Conditional Possibility Tables (CPT) of an existing possibilistic network into uncertain gates. In possibilistic networks, CPT parameters must be elicited by an expert but when the number of parents of a variable grows, the number of parameters to elicit grows exponentially.	This problem generates difficulties for experts to elicit all parameters because it is time-consuming. One solution consists in using uncertain gates to compute automatically CPTs. This is useful in knowledge engineering. When possibilistic networks already exist, it can be interesting to transform them by using uncertain gates because we can highlight the  combination behaviour of the variables. To illustrate our approach, we will present at first a simple example of the estimation for 3 test CPTs with behaviours MIN, MAX and weighted average. Then, we will perform a more significant experimentation which will consist in converting a set of Bayesian networks into possibilistic networks to perform the estimation of CPTs by uncertain gates. 

\keywords{Possibilistic networks, Possibility theory, Uncertain logical gates, Estimation.}



\end{document}
