\documentclass[../booklet.tex]{subfiles}

\begin{document}

\subsection[Dealing with inconsistent measurements in inverse problems: an approach based on sets and intervals.. {\it Krushna Shinde, Pierre Feissel and Sébastien Destercke}]{Dealing with inconsistent measurements in inverse problems: an approach based on sets and intervals.}
\index[authors]{Shinde, Krushna} \index[authors]{Feissel, Pierre} \index[authors]{Destercke, Sébastien}

\begin{center}
  {\it Krushna Shinde, Pierre Feissel and Sébastien Destercke}
\end{center}
%\begin{minipage}{1\textwidth}
%\end{minipage}

\vskip 0.8cm


We consider the (inverse) problem of finding back the parameter values of a physical model given a set of measurements. As the deterministic solution to this problem is sensitive to measurement error in the data, one way to resolve this issue is to take into account uncertainties in the data. In this paper, we explore how interval-based approaches can be used to obtain a solution to the inverse problem, in particular when measurements are inconsistent with one another. We show on a set of experiments, in which we compare the set-based approach with the Bayesian one, that this is particularly interesting when some measurements can be suspected of being outliers.  
\keywords{Inverse problem  \and Interval uncertainty \and Outlier detection.}



\end{document}
