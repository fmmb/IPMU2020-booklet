\documentclass[../booklet.tex]{subfiles}

\begin{document}

\subsection[On Ruspini's models of similarity-based approximate reasoning. {\it Francesc Esteva, Lluis Godo, Ricardo Oscar Rodriguez and Thomas Vetterlein}]{On Ruspini's models of similarity-based approximate reasoning}
\index[authors]{Esteva, Francesc} \index[authors]{Godo, Lluis} \index[authors]{Rodriguez, Ricardo Oscar} \index[authors]{Vetterlein, Thomas}

\begin{center}
  {\it Francesc Esteva, Lluis Godo, Ricardo Oscar Rodriguez and Thomas Vetterlein}
\end{center}
%\begin{minipage}{1\textwidth}
%\end{minipage}

%\vskip 0.8cm


%
In his 1991 seminal paper, Enrique H.\ Ruspini proposed a similarity-based semantics for fuzzy sets and approximate reasoning which has been extensively used by many other authors in various contexts. This brief note, which is our humble contribution to honor Ruspini's great legacy, describes some of the main developments in the field of logic that essentially rely on his ideas.
%
\keywords{Fuzzy similarity, Approximate reasoning, Graded entailments, Modal logic}



\end{document}
