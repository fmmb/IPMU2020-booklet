\documentclass[../booklet.tex]{subfiles}

\begin{document}

\section[Modus Ponens Tollens for RU-implications. {\it Isabel Aguiló, Sebastià Massanet, Juan Vicente Riera and Daniel Ruiz-Aguilera}]{Modus Ponens Tollens for RU-implications}
\index[authors]{Aguiló, Isabel} \index[authors]{Massanet,  Sebastià} \index[authors]{Riera,  Juan Vicente} \index[authors]{Ruiz-Aguilera, Daniel}

\begin{center}
  {\it Isabel Aguiló, Sebastià Massanet, Juan Vicente Riera and Daniel Ruiz-Aguilera}
\end{center}
%\begin{minipage}{1\textwidth}
%\end{minipage}

\vskip 0.8cm


In fuzzy rules based systems, fuzzy implication functions are usually considered to model fuzzy conditionals and to perform forward and backward inferences. These processes are guaranteed by the fulfilment of the Modus Ponens and Modus Tollens properties by the fuzzy implication function with respect to the considered conjunction and fuzzy negation. In this paper, we investigate which residual implications derived from uninorms satisfy both Modus Ponens and Modus Tollens properties with respect to the same t-norm and a fuzzy negation simultaneously. The most usual classes of uninorms are considered and many solutions are obtained which allow to model the fuzzy conditionals in a fuzzy rules based systems (and perform backward and forward inferences) with a unique residual implication derived from a uninorm.  
\keywords{Fuzzy implication Function \and Modus Ponens \and Modus Tollens \and Uninorm.}



\end{document}
