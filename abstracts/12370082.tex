\documentclass[../booklet.tex]{subfiles}

\begin{document}

\subsection[A  L1 minimization optimal corrective explanation procedure for probabilistic databases. {\it Marco Baioletti and Andrea Capotorti}]{A  L1 minimization optimal corrective explanation procedure for probabilistic databases}
\index[authors]{Baioletti, Marco} \index[authors]{Capotorti, Andrea}

\begin{center}
  {\it Marco Baioletti and Andrea Capotorti}
\end{center}
%\begin{minipage}{1\textwidth}
%\end{minipage}

%\vskip 0.8cm


We propose to use a, recently introduced,  efficient $L1$ distance minimization through mixed-integer
linear programming for minimizing the number of valuations to be modified inside an incoherent probabilistic assessment. This is in line with one basic principle of optimal corrective explanation for decision makers.

 A shrewd use of constraints and of slack variables permit to steer the correction of incoherent assessments towards aimed directions, like e.g. the minimal number of changes. Such corrective explanations can be searched alone, as minimal changes, or jointly with the property of being also inside the $L1$ distance minimizers (in a bi-optimal point of view).
 
 The detection of such bi-optimal solutions can be performed efficiently by profiting from the geometric characterization of the whole set of $L1$ minimizers and from the properties of $L1$ topology.
 
\keywords{Incoherence corrections, $L1$ constrained minimization, mixed integer programming, optimal corrective explanation, probabilistic databases}



\end{document}
