\documentclass[../booklet.tex]{subfiles}

\begin{document}

\section[Covariate-Adjusted Hybrid Principal Components Analysis. {\it Aaron Scheffler, Abigail Dickinson, Charlotte DiStefano, Shafali Jeste and Damla Senturk}]{Covariate-Adjusted Hybrid Principal Components Analysis}
\index[authors]{Scheffler, Aaron} \index[authors]{Dickinson,  Abigail} \index[authors]{DiStefano,  Charlotte} \index[authors]{Jeste,  Shafali} \index[authors]{Senturk, Damla}

\begin{center}
  {\it Aaron Scheffler, Abigail Dickinson, Charlotte DiStefano, Shafali Jeste and Damla Senturk}
\end{center}
%\begin{minipage}{1\textwidth}
%\end{minipage}

\vskip 0.8cm


Electroencephalography (EEG) studies produce region-referenced functional data in the form of EEG signals recorded across electrodes on the scalp. The high-dimensional data capture underlying neural dynamics and it is of clinical interest to model differences in neurodevelopmental trajectories between diagnostic groups, for example typically developing (TD) children and children with autism spectrum disorder (ASD). In such cases, valid group-level inference requires characterization of the complex EEG dependency structure as well as covariate-dependent heteroscedasticity, such as changes in variation over developmental age. In our motivating study, resting state EEG is collected on both TD and ASD children aged two to twelve years old. The peak alpha frequency (PAF), defined as the location of a prominent peak in the alpha frequency band of the spectral density, is an important biomarker linked to neurodevelopment and is known to shift from lower to higher frequencies as children age. To retain the most amount of information from the data, we model patterns of alpha spectral variation, rather than just the peak location, regionally across the scalp and chronologically across development for both the TD and ASD diagnostic groups. We propose a covariate-adjusted hybrid principal components analysis (CA-HPCA) for region-referenced functional EEG data, which utilizes both vector and functional principal components analysis while simultaneously adjusting for covariate-dependent heteroscedasticity. CA-HPCA assumes the covariance process is weakly separable conditional on observed covariates, allowing for covariate-adjustments to be made on the marginal covariances rather than the full covariance leading to stable and computationally efficient estimation. A mixed effects framework is proposed to estimate the model components coupled with a bootstrap test for group-level inference. The proposed methodology provides novel insights into neurodevelopmental differences between TD and ASD children. 

\keywords{Electroencephalography; Autism Spectrum Disorder; Functional data analysis; Marginal covariances; Functional principal components analysis; Covariate-adjustments}




\end{document}
