\documentclass[../booklet.tex]{subfiles}

\begin{document}

\subsection[Intelligent Detection of Information Outliers using Linguistic Summaries \\with Non-monotonic Quantifiers. {\it Agnieszka Duraj, Piotr S. Szczepaniak and Łukasz Chomątek}]{Intelligent Detection of Information Outliers using Linguistic Summaries \\with Non-monotonic Quantifiers}
\index[authors]{Duraj, Agnieszka} \index[authors]{Szczepaniak,  Piotr S.} \index[authors]{Chomątek, Łukasz}

\begin{center}
  {\it Agnieszka Duraj, Piotr S. Szczepaniak and Łukasz Chomątek}
\end{center}
%\begin{minipage}{1\textwidth}
%\end{minipage}

\vskip 0.8cm


In the processing of imprecise information, principally in big data analysis, it is very advantageous to transform numerical values into the standard form of linguistic statements. 
This paper deals with a novel method of outlier detection using linguistic summaries. Particular attention is devoted to examining the usefulness of non-monotonic quantifiers, which represent a fuzzy determination of the amount of analyzed data.The answer is positive. The use of non-monotonic quantifiers in the detection of outliers can provide a more significant value of the degree of truth of a linguistic summary. At the end, this paper provides a computational example of practical importance.

\keywords{intelligent data analysis, linguistic summaries, monotonic and non-monotonic quantifiers, intelligent outlier detection}



\end{document}
