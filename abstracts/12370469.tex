\documentclass[../booklet.tex]{subfiles}

\begin{document}

\section[Graphical Causal Models and Imputing Missing Data: A Preliminary Study. {\it Rui Jorge Almeida, Greetje Adriaans and Yuliya Shapovalova}]{Graphical Causal Models and Imputing Missing Data: A Preliminary Study}
\index[authors]{Almeida, Rui Jorge} \index[authors]{Adriaans,  Greetje} \index[authors]{Shapovalova, Yuliya}

\begin{center}
  {\it Rui Jorge Almeida, Greetje Adriaans and Yuliya Shapovalova}
\end{center}
%\begin{minipage}{1\textwidth}
%\end{minipage}

\vskip 0.8cm


Real-world datasets often contain many missing values due to several reasons. This is usually an issue since many learning algorithms require complete datasets. In certain cases, there are constraints in the real world problem that create difficulties in continuously observing all data. In this paper, we investigate if graphical causal models can be used to impute missing values and derive additional information on the uncertainty of the imputed values. Our goal is to use the information from a complete dataset in the form of graphical causal models to impute missing values in an incomplete dataset. This assumes that the datasets have the same data generating process. Furthermore, we calculate the probability of each missing data value belonging to a specified percentile. We present a preliminary study on the proposed method using synthetic data, where we can control the causal relations and missing values.
\keywords{Missing Data  \and Graphical Causal Models \and Uncertainty in Missing Values}



\end{document}
