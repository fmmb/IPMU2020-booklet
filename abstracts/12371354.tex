\documentclass[../booklet.tex]{subfiles}

\begin{document}

\subsection[A Genetic Approach to the Job Shop Scheduling Problem with Interval Uncertainty. {\it Hernán Díaz, Inés González-Rodríguez, Juan José Palacios, Irene Díaz and Camino R. Vela}]{A Genetic Approach to the Job Shop Scheduling Problem with Interval Uncertainty}
\index[authors]{Díaz, Hernán} \index[authors]{González-Rodríguez, Inés} \index[authors]{Palacios, Juan José} \index[authors]{Díaz, Irene} \index[authors]{Vela, Camino R.}

\begin{center}
  {\it Hernán Díaz, Inés González-Rodríguez, Juan José Palacios, Irene Díaz and Camino R. Vela}
\end{center}
%\begin{minipage}{1\textwidth}
%\end{minipage}

%\vskip 0.8cm


In this paper we tackle a variant of the job shop scheduling problem where task durations are uncertain and only an interval of possible values for each task duration is known. We propose a genetic algorithm to minimise the schedule's makespan that takes into account the problem's uncertainty during the search process. The behaviour of the algorithm is experimentally evaluated  and compared with other state-of-the-art algorithms. Further analysis in terms of solution robustness proves the advantage of taking into account interval uncertainty during the search process with respect to considering only the expected processing times and solving the problem's crisp counterpart. This robustness analysis also illustrates the relevance of the interval ranking method used to compare schedules during the search.
\keywords{Job shop scheduling, {Interval processing time} \and Genetic algorithms \and Robustness}



\end{document}
