\documentclass[../booklet.tex]{subfiles}

\begin{document}

\subsection[A Comparison of Explanatory Measures in Abductive Inference. {\it Jiandong Huang, David Glass and Mark Mccartney}]{A Comparison of Explanatory Measures in Abductive Inference}
\index[authors]{Huang, Jiandong} \index[authors]{Glass, David} \index[authors]{Mccartney, Mark}

\begin{center}
  {\it Jiandong Huang, David Glass and Mark Mccartney}
\end{center}
%\begin{minipage}{1\textwidth}
%\end{minipage}

%\vskip 0.8cm

Computer simulations have been carried out to investigate the performance of two measures for abductive inference, Maximum Likelihood (ML), and Product Coherence Measure (PCM), by comparing them with a third approach, Most Probable Explanation (MPE). These have been realized through experiments that compare outcomes from a specified model (the correct model) with those from incorrect models which assume that the hypotheses are mutually exclusive or independent. The results show that PCM tracks the results of MPE more closely than ML when the degree of competition is greater than 0 and hence is able to infer explanations that are more likely to be true under such a condition. Experiments on the robustness of the measures with respect to incorrect model assumptions show that ML is more robust in general, but that MPE and PCM are more robust when the degree of competition is positive.  The results also show that in general it is more reasonable to assume the hypotheses in question are independent than to assume they are mutually exclusive. 

\keywords{Inference to the Best Explanation (IBE). Explanatory reasoning. Hypotheses competition. Abduction}




\end{document}
