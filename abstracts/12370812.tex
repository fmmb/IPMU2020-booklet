\documentclass[../booklet.tex]{subfiles}

\begin{document}

\subsection[Cautious label-wise ranking with constraint satisfaction. {\it Yonatan Carlos Carranza Alarcon, Soundouss Messoudi and Sébastien Destercke}]{Cautious label-wise ranking with constraint satisfaction}
\index[authors]{Alarcon, Yonatan Carlos Carranza} \index[authors]{Messoudi, Soundouss} \index[authors]{Destercke, Sébastien}

\begin{center}
  {\it Yonatan Carlos Carranza Alarcon, Soundouss Messoudi and Sébastien Destercke}
\end{center}
%\begin{minipage}{1\textwidth}
%\end{minipage}

%\vskip 0.8cm


Ranking problems are difficult to solve due to their combinatorial nature. One way to solve this issue is to adopt a decomposition scheme, splitting the initial difficult problem in many simpler problems. The predictions obtained from these simplified settings must then be combined into one single output, possibly resolving inconsistencies between the outputs. In this paper, we consider such an approach for the label ranking problem, where in addition we allow the predictive model to produce cautious inferences in the form of sets of rankings when it lacks information to produce reliable, precise predictions. More specifically, we propose to combine a rank-wise decomposition, in which every sub-problem becomes an ordinal classification one, with a constraint satisfaction problem (CSP) approach to verify the consistency of the predictions. Our experimental results indicate that our approach produces predictions with appropriately balanced reliability and precision, while remaining competitive with classical, precise approaches.
\keywords{Label ranking problem, Constraint satisfaction, Imprecise probabilities.}



\end{document}
