\documentclass[../booklet.tex]{subfiles}

\begin{document}

\section[Impact of local congruences in attribute reduction. {\it Roberto G. Aragón, Jesús Medina and Eloísa Ramírez-Poussa}]{Impact of local congruences in attribute reduction}
\index[authors]{Aragón, Roberto G.} \index[authors]{Medina,  Jesús} \index[authors]{Ramírez-Poussa, Eloísa}

\begin{center}
  {\it Roberto G. Aragón, Jesús Medina and Eloísa Ramírez-Poussa}
\end{center}
%\begin{minipage}{1\textwidth}
%\end{minipage}

\vskip 0.8cm


Local congruences are equivalence relations whose equivalence classes are convex sublattices of the original lattice. In this paper, we present a study that relates local congruences to attribute reduction in FCA. Specifically,   we will analyze the impact in the context of the use of local congruences, when they are used  for complementing an attribute reduction.

%Local congruence are equivalence relations whose equivalence classes are convex sublattices of the original lattice. In this paper, we present a study that relates local congruences to attribute reductions in FCA. Specifically, in this work we analyze when  the equivalence relation induced by  an attribute reduction of a formal context is a local congruence. 

%This paper addresses the problem of {attribute and} size reduction of concept lattices in formal concept analysis. The reduction of the number of attributes in a formal context produces a partition on the set of concepts of the concept lattice. In this work, we introduce a weaker notion of congruence relation, called  local congruence. This less restrictive kind of congruence   guarantees that each subset of the  partition forms a closed algebraic substructure, aggregating as few concepts as possible and preserving the main information.
\keywords{formal concept analysis, size reduction, attribute reduction, local congruence}



\end{document}
