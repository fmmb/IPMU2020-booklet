\documentclass[../booklet.tex]{subfiles}

\begin{document}

\subsection[Ordinal Graph-based Games. {\it Arij Azzabi, Nahla Ben Amor, Hélène Fargier and Régis Sabbadin}]{ Ordinal Graph-based Games}
\index[authors]{Azzabi, Arij} \index[authors]{Amor,  Nahla Ben} \index[authors]{Fargier,  Hélène} \index[authors]{Sabbadin, Régis}

\begin{center}
  {\it Arij Azzabi, Nahla Ben Amor, Hélène Fargier and Régis Sabbadin}
\end{center}
%\begin{minipage}{1\textwidth}
%\end{minipage}

\vskip 0.8cm

 
 The graphical, hypergraphical and polymatrix games frameworks provide concise representations of non-cooperative normal-form games involving many agents. In these {\em graph-based} games, agents interact in simultaneous local subgames with the agents which are their neighbors in a graph. 
Recently, ordinal normal form games have been proposed as a framework for game theory where agents' utilities are ordinal. 
This paper presents the first definition of \emph{Ordinal Graphical Games} (OGG),  \emph{Ordinal Hypergraphical Games} (OHG), and \emph{Ordinal Polymatrix Games} (OPG). 
We show that, as for classical graph-based games, determining whether a pure NE exists is also NP-hard. We propose an original CSP model to decide their existence and compute them.
Then, a polynomial-time algorithm to compute possibilistic mixed equilibria for graph-based games is proposed. Finally, the experimental study is dedicated to test our proposed solution concepts for ordinal graph-based games.

{\bf Keywords:} Possibility theory, ordinal game theory, algorithms and complexity 

\end{document}
