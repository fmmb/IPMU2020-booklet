\documentclass[../booklet.tex]{subfiles}

\begin{document}

\section[General grouping functions. {\it Helida Santos, Graçaliz P. Dimuro, Tiago C. Asmus, Giancarlo Lucca, Eduardo N. Borges, Benjamin Bedregal, José A. Sanz, Javier Fernández and Humberto Bustince}]{General grouping functions}
\index[authors]{Santos, Helida} \index[authors]{Dimuro,  Graçaliz P.} \index[authors]{Asmus,  Tiago C.} \index[authors]{Lucca,  Giancarlo} \index[authors]{Borges,  Eduardo N.} \index[authors]{Bedregal,  Benjamin} \index[authors]{Sanz,  José A.} \index[authors]{Fernández,  Javier} \index[authors]{Bustince, Humberto}

\begin{center}
  {\it Helida Santos, Graçaliz P. Dimuro, Tiago C. Asmus, Giancarlo Lucca, Eduardo N. Borges, Benjamin Bedregal, José A. Sanz, Javier Fernández and Humberto Bustince}
\end{center}
%\begin{minipage}{1\textwidth}
%\end{minipage}

\vskip 0.8cm


Some aggregation functions that are not necessarily associative, namely overlap and grouping functions, have called the attention of many researchers in the recent past. This is probably due to the fact that they are a richer class of operators whenever one compares with other classes of aggregation functions, such as t-norms and t-conorms, respectively. In the present work we introduce a more general proposal for disjunctive $n$-ary aggregation functions entitled general grouping functions, in order to be used in problems that admit $n$ dimensional inputs in a more flexible manner, allowing their application in different contexts. We present some new interesting results, like the characterization of that operator and also provide different construction methods.

\keywords{Grouping functions \and $n$-dimensional grouping functions  \and General grouping functions \and General overlap functions.}



\end{document}
