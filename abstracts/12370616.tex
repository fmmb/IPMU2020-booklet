\documentclass[../booklet.tex]{subfiles}

\begin{document}

\subsection[Modeling the Costs of Trade Finance during the Financial Crisis of 2008-2009: An Application of Dynamic Hierarchical Linear Model. {\it Shantanu Mullick, Ashwin Malshe and Nicolas Glady}]{Modeling the Costs of Trade Finance during the Financial Crisis of 2008-2009: An Application of Dynamic Hierarchical Linear Model}
\index[authors]{Mullick, Shantanu} \index[authors]{Malshe, Ashwin} \index[authors]{Glady, Nicolas}

\begin{center}
  {\it Shantanu Mullick, Ashwin Malshe and Nicolas Glady}
\end{center}
%\begin{minipage}{1\textwidth}
%\end{minipage}

\vskip 0.8cm

The authors propose a dynamic hierarchical linear model (DHLM) to study the variations in the costs of trade finance over time and across countries in dynamic environments such as the global financial crisis of 2008-2009. The DHLM can cope with challenges that a dynamic environment entails: nonstationarity, parameters changing over time and cross-sectional heterogeneity. The authors employ a DHLM to examine how the effects of four macroeconomic indicators -- GDP growth, inflation, trade intensity and stock market capitalization - on trade finance costs varied over a period of five years from 2006 to 2010 across 8 countries. We find that the effect of these macroeconomic indicators varies over time, and most of this variation is present in the year preceding and succeeding the financial crisis. In addition, the trajectory of time-varying effects of GDP growth and inflation support the "flight to quality" hypothesis: cost of trade finance reduces in countries with high GDP growth and low inflation, during the crisis. The authors also note presence of country-specific heterogeneity in some of these effects. The authors propose extensions to the model and discuss its alternative uses in different contexts.

\keywords{Trade Finance, Financial Crisis, Bayesian Methods, Time Series Analysis}


\end{document}
