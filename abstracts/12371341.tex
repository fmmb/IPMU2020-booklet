\documentclass[../booklet.tex]{subfiles}

\begin{document}

\section[Mining text patterns over fake and real tweets. {\it J. Angel Diaz-Garcia, Carlos Fernandez-Basso, M. Dolores Ruiz and Maria J. Martin-Bautista}]{Mining text patterns over fake and real tweets}
\index[authors]{Diaz-Garcia, J. Angel} \index[authors]{Fernandez-Basso,  Carlos} \index[authors]{Ruiz,  M. Dolores} \index[authors]{Martin-Bautista, Maria J.}

\begin{center}
  {\it J. Angel Diaz-Garcia, Carlos Fernandez-Basso, M. Dolores Ruiz and Maria J. Martin-Bautista}
\end{center}
%\begin{minipage}{1\textwidth}
%\end{minipage}

\vskip 0.8cm


	
With the exponential growth of users and user-generated content present on online social networks, fake  news and its detection have become a major problem. Through these, smear campaigns can be generated, aimed for example at trying to change the political orientation of some people. Twitter has become one of the main spreaders of fake news in the network. Therefore, in this paper, we present a solution based on Text Mining that tries to find which text patterns are related to tweets that refer to fake news and which patterns in the tweets are related to true news. To test and validate the results, the system faces a pre-labelled dataset of fake and real tweets during the U.S. presidential election in 2016. In terms of results interesting patterns are obtained that relate the size and subtle changes of the real news to create fake news. Finally, different ways to visualize the results are provided. 	

\keywords{Association Rules, social media mining, fake news, Text Mining, Twitter}



\end{document}
