\documentclass[../booklet.tex]{subfiles}

\begin{document}

\subsection[Two dualities for weakly pseudo-complemented quasi-Kleene algebras. {\it Umberto Rivieccio, Ramon Jansana and Thiago N. Silva}]{Two dualities for weakly pseudo-complemented quasi-Kleene algebras}
\index[authors]{Rivieccio, Umberto} \index[authors]{Jansana,  Ramon} \index[authors]{Silva, Thiago N.}

\begin{center}
  {\it Umberto Rivieccio, Ramon Jansana and Thiago N. Silva}
\end{center}
%\begin{minipage}{1\textwidth}
%\end{minipage}

\vskip 0.8cm


Quasi-Nelson algebras are a non-involutive generalisation of Nelson algebras that
%This class
can be %has been recently singled out and 
characterised in several %equivalent 
ways,
e.g.~as (i) the variety of bounded commutative integral (not necessarily involutive)  residuated lattices that satisfy the Nelson identity; %as well as 
(ii) the class of $(0,1)$-congruence orderable commutative integral  residuated lattices; %or
(iii) the algebraic counterpart of quasi-Nelson logic, i.e.~the 
(algebraisable) extension of the substructural logic
$\mathcal{FL}_{ew}$ %(Full Lambek calculus with Exchange and Weakening) 
by the Nelson axiom. 
%Quasi-Nelson logic may also be viewed as a common
%generalisation of both Nelson's constructive logic with strong negation and intuitionistic logic. In the present paper, 
In the present paper  we focus on the subreducts of quasi-Nelson algebras
obtained by eliding the implication while keeping the two term-definable negations.
These form a variety %of algebras 
that 
(following A.~Sendlewski, who studied the corresponding fragment of Nelson algebras)
we dub \emph{weakly pseudo-complemented quasi-Kleene algebras}.
We develop a Priestley-style duality  for these algebras (in two different guises)
which is essentially an application of the %two-sorted 
general approach %to dualities 
proposed in the paper \emph{A duality for two-sorted lattices}
by A.~Jung and U.~Rivieccio.
%, that is based on the twist-algebra
%representation introduced by the first author in 
%We show that, similarly to the involutive case (studied by A.~Sendlewski), this class of algebras is a variety that can be characterised
%by means of twist-algebras over pseudo-complemented distributive lattices.  In this way, we extend to a non-involutive setting the well-known  connection between
%Nelson and Heyting algebras, as well as Sendlewski's  result that relates Kleene algebras with a weak pseudo-complementation and pseudo-complemented distributive lattices.


\keywords{Quasi-Nelson algebras \and Kleene algebras with weak pseudo-complement
 \and Residuated lattices  \and
 Two-sorted duality  \and Semi-De Morgan.}



\end{document}
