\documentclass[../booklet.tex]{subfiles}

\begin{document}

\subsection[A computational study on the entropy of interval-valued datasets from the stock market. {\it Chenyi Hu and Zhihui Hu}]{A computational study on the entropy of interval-valued datasets from the stock market}
\index[authors]{Hu, Chenyi} \index[authors]{Hu, Zhihui}

\begin{center}
  {\it Chenyi Hu and Zhihui Hu}
\end{center}
%\begin{minipage}{1\textwidth}
%\end{minipage}

\vskip 0.8cm


Using interval-valued data and computing, researchers have reported significant quality improvements of the stock market annual variability forecasts recently. Through studying the entropy of interval-valued datasets, this work provides both information theoretic and empirical evidences on that the significant quality improvements are very likely come from interval-valued datasets.  
Therefore, using interval-valued samples rather than point-valued ones is preferable in making variability forecasts. 
This study also computationally investigates the impacts of data aggregation methods and probability distributions on the entropy of interval-valued datasets. Computational results suggest that both min-max and confidence intervals can work well in aggregating point-valued data into intervals. However, assuming uniform probability distribution should be a good practical choice in calculating the entropy of an interval-valued dataset in some applications at least. % We further investigates the impacts of different  interval data generation strategies on information entropy for a possible preferred selection. From a perspective of information theory, this study explains . 



\end{document}
