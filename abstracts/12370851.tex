\documentclass[../booklet.tex]{subfiles}

\begin{document}

\subsection[Conditioning and Dilation with Coherent Nearly-Linear Models. {\it Renato Pelessoni and Paolo Vicig}]{Conditioning and Dilation with Coherent Nearly-Linear Models}
\index[authors]{Pelessoni, Renato} \index[authors]{Vicig, Paolo}

\begin{center}
  {\it Renato Pelessoni and Paolo Vicig}
\end{center}
%\begin{minipage}{1\textwidth}
%\end{minipage}

%\vskip 0.8cm


In previous work [REF] we introduced Nearly-Linear (NL) models, a class of neighbourhood models obtaining upper/lower probabilities by means of a linear affine transformation (with barriers) of a given probability.
NL models are partitioned into more subfamilies, some of which are coherent.
One, that of the Vertical Barrier Models (VBM), includes known models, such as the Pari-Mutuel, the $\varepsilon$-contamination or the Total Variation model as special instances.
In this paper we study conditioning of coherent NL models, obtaining formulae for their natural extension.
We show that VBMs are stable after conditioning, i.e. return a conditional model that is still a VBM, and that this is true also for the special instances mentioned above but not in general for NL models. We then analyse dilation for coherent NL models, a phenomenon that makes our \emph{ex-post} opinion on an event $A$, after conditioning it on any event in a partition of hypotheses, vaguer than our \emph{ex-ante} opinion on $A$.

\keywords{Conditioning, Coherent imprecise probabilities, Nearly-Linear models, Dilation.}



\end{document}
