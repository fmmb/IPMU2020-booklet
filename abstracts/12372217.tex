\documentclass[../booklet.tex]{subfiles}

\begin{document}

\subsection[Towards a classification of rough set bireducts. {\it M. José Benítez-Caballero, Jesús Medina and Eloisa Ramírez Poussa}]{Towards a classification of rough set bireducts}
\index[authors]{Benítez-Caballero, M. José} \index[authors]{Medina,  Jesús} \index[authors]{Poussa, Eloisa Ramírez}

\begin{center}
  {\it M. José Benítez-Caballero, Jesús Medina and Eloisa Ramírez Poussa}
\end{center}
%\begin{minipage}{1\textwidth}
%\end{minipage}

\vskip 0.8cm


Size reduction mechanisms are very important  in several mathematical fields. In rough set theory, bireducts arose to reduce  simultaneously  the set of attributes and the set of objects of the considered dataset, providing subsystems with the minimal sets of attributes that connect the maximum number of objects preserving the information of the original dataset. This paper   presents the main properties of bireducts and how they can be used  for removing inconsistencies. 
\keywords{Rough Set Theory, Bireducts, Size Reduction.}



\end{document}
