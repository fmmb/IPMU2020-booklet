\documentclass[../booklet.tex]{subfiles}

\begin{document}

\subsection[Data-Driven Classifiers for Predicting Grass Growth in Northern Ireland: A Case Study. {\it Orla McHugh, Jun Liu, Fiona Browne, Philip Jordan and Deborah McConnell}]{Data-Driven Classifiers for Predicting Grass Growth in Northern Ireland: A Case Study}
\index[authors]{McHugh, Orla} \index[authors]{Liu,  Jun} \index[authors]{Browne,  Fiona} \index[authors]{Jordan,  Philip} \index[authors]{McConnell, Deborah}

\begin{center}
  {\it Orla McHugh, Jun Liu, Fiona Browne, Philip Jordan and Deborah McConnell}
\end{center}
%\begin{minipage}{1\textwidth}
%\end{minipage}

\vskip 0.8cm

There are increasing pressures to combat climate change and improve sustainable land management. The agriculture industry is one of the most challenging areas for these changes, especially in Northern Ireland, as agriculture is one of the larger industries. Research has been carried out across the island of Ireland into methods of improving farm efficiency in multiple areas of farming, including livestock health, machinery improvements, and crop growth. Research has been carried out in this study into grass growth in the dairy farming sector, specifically within Northern Ireland. Grass growth prediction aims to inform farmers and policy makers in their decision-making process regarding sustainable land management in agriculture. The present work focuses on analysing and evaluating how data-driven classifiers can be used for grass growth prediction using the data related to soil content, weather, grass quality components etc. Four classifiers, namely Decision Trees, Random Forest, Naïve Bayes, and Neural Networks, are chosen for this purpose. Classification results based on a real-world data set are analysed and compared to evaluate and illustrate the performance and robustness of the classifiers. The results indicate that it is difficult to declare a single classifier with the highest performance and robustness. Nevertheless, it indicates that tree classification methods are better suited to the data to be studied, as opposed to probabilistic methods and weighted methods, e.g., the naïve Bayes classifier obtained a predictive performance of 78% when classifying spring seasonal grass growth data.
\keywords{Climate Change, Grass Growth Prediction, Data-driven Classifier}


\end{document}
