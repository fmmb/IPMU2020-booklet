\documentclass[../booklet.tex]{subfiles}

\begin{document}

\subsection[Hybrid model for Parkinson's disease prediction. {\it Augusto Junio Guimarães, Paulo Vitor Campos Souza and Edwin Lughofer}]{Hybrid model for Parkinson's disease prediction}
\index[authors]{Guimarães, Augusto Junio} \index[authors]{Souza, Paulo Vitor Campos} \index[authors]{Lughofer, Edwin}

\begin{center}
  {\it Augusto Junio Guimarães, Paulo Vitor Campos Souza and Edwin Lughofer}
\end{center}
%\begin{minipage}{1\textwidth}
%\end{minipage}

\vskip 0.8cm


Parkinson's is a chronic, progressive neurological disease with no known cause that affects the central nervous system of older people and compromises their movement. This disorder can impair daily aspects of people and therefore identify their existence early, helps in choosing treatments that can reduce the impact of the disease on the patient's routine. This work aims to identify Parkinson's traces through a voice recording replications database applied to a fuzzy neural network to identify their patterns and enable the extraction of knowledge about situations present in the data collected in patients. The results obtained by the hybrid model were superior to state of the art for the theme, proving that it is possible to perform hybrid models in the extraction of knowledge and the classification of behavioral patterns of high accuracy Parkinson's.

\keywords{Parkinson's Disease  \and Fuzzy Neural Network \and Hybrid Models.}



\end{document}
