\documentclass[../booklet.tex]{subfiles}

\begin{document}

\subsection[Random Steinhaus distances for robust syntax-based classification of partially inconsistent linguistic data. {\it Laura Franzoi, Andrea Sgarro, Anca Dinu and Liviu P. Dinu}]{Random Steinhaus distances for robust syntax-based classification of partially inconsistent linguistic data}
\index[authors]{Franzoi, Laura} \index[authors]{Sgarro, Andrea} \index[authors]{Dinu, Anca} \index[authors]{Dinu, Liviu P.}

\begin{center}
  {\it Laura Franzoi, Andrea Sgarro, Anca Dinu and Liviu P. Dinu}
\end{center}
%\begin{minipage}{1\textwidth}
%\end{minipage}

%\vskip 0.8cm


We use the Steinhaus transform of metric distances to deal with inconsistency in linguistic classification.  We focus on data due to G.\ Longobardi's school: languages are represented through yes-no strings of length 53, each string position corresponding to a syntactic feature which can be present or absent. However, due to a complex network of logical implications which constrain features, some positions might be undefined (logically inconsistent). To take into account linguistic inconsistency, the distances we use are Steinhaus metric distances generalizing the normalized Hamming distance. To validate the robustness of classifications based on Longobardi's data we resort to randomized transforms. Experimental results are provided and commented upon.

\keywords{Steinhaus distance, linguistic classification, L ukasiewicz logic, fuzzy logic}

\end{document}
