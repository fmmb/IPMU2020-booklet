\documentclass[../booklet.tex]{subfiles}

\begin{document}

\subsection[The Formalization of Asymmetry in Disjunctive Evaluation. {\it Miroslav Hudec and Radko Mesiar}]{The Formalization of Asymmetry in Disjunctive Evaluation}
\index[authors]{Hudec, Miroslav} \index[authors]{Mesiar, Radko}

\begin{center}
  {\it Miroslav Hudec and Radko Mesiar}
\end{center}
%\begin{minipage}{1\textwidth}
%\end{minipage}

\vskip 0.8cm


The main property of disjunction is substitutability, i.e., the fully satisfied predicate substitutes the rejected one. But, in many real--world cases disjunction is expressed as the fusion of full and optional alternatives, which is expressed as \textit{OR ELSE} connective. 
Generally, this logical connective provides a solution lower than or equal to the \textit{MAX} operator, and higher than or equal to the projection of the full alternative, i.e., the solution does not go below any averaging function and above \textit{MAX} function. Therefore, the optional alternative does not influence the solution when it is satisfied with a degree lower than the degree of full alternative. In this work, we propose further generalization by other disjunctive functions in order to allow upward reinforcement of asymmetric disjunction. Finally, the obtained results are illustrated and discussed.

\keywords{Asymmetric disjunction \and Averaging functions \and Probabilistic sum \and \L{}ukasiewicz t--conorm \and Generalization \and Upward reinforcement.}



\end{document}
