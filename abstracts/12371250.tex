\documentclass[../booklet.tex]{subfiles}

\begin{document}

\section[Analyzing non-deterministic computable aggregations. {\it Luis Magdalena, Luis Garmendia, Daniel Gómez and Javier Montero}]{Analyzing non-deterministic computable aggregations}
\index[authors]{Magdalena, Luis} \index[authors]{Garmendia,  Luis} \index[authors]{Gómez,  Daniel} \index[authors]{Montero, Javier}

\begin{center}
  {\it Luis Magdalena, Luis Garmendia, Daniel Gómez and Javier Montero}
\end{center}
%\begin{minipage}{1\textwidth}
%\end{minipage}

\vskip 0.8cm



Traditionally, the term aggregation is associated with an aggregation function, implicitly assuming that any aggregation process can be represented by a function. However, the concept of computable aggregation considers that the core of the aggregation processes is the program that enables it. This new concept of aggregation introduces the scenario where the aggregation can even be non-deterministic. In this work, this new class of aggregation is formally defined, and some desirable properties related with consistency, robustness and monotonicity are proposed.  

\keywords{Aggregation  \and Computable aggregation \and nondeterministic aggregation}



\end{document}
