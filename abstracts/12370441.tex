\documentclass[../booklet.tex]{subfiles}

\begin{document}

\subsection[A Probabilistic Approach for Discovering Daily Human Mobility Patterns with Mobile Data. {\it Weizhu Qian, Fabrice Lauri and Franck Gechter}]{A Probabilistic Approach for Discovering Daily Human Mobility Patterns with Mobile Data}
\index[authors]{Qian, Weizhu} \index[authors]{Lauri, Fabrice} \index[authors]{Gechter, Franck}

\begin{center}
  {\it Weizhu Qian, Fabrice Lauri and Franck Gechter}
\end{center}
%\begin{minipage}{1\textwidth}
%\end{minipage}

\vskip 0.8cm


Analyzing human mobility with geo-location data collected from smartphones has been a hot research topic in recent years. In this paper, we attempt to discover daily mobile patterns using the GPS data. In particular, we view this problem from a probabilistic perspective. A non-parametric Bayesian modeling method, the Infinite Gaussian Mixture Model (IGMM) is used to estimate the probability density of the daily mobility. We also utilize the Kullback-Leibler (KL) divergence as the metrics to measure the similarity of different probability distributions. Combining the IGMM and the KL divergence, we propose an automatic clustering algorithm to discover mobility patterns for each individual user. Finally, the effectiveness of our method is validated on the real user data collected from different real users. 

\keywords{probabilistic model \and infinite Gaussian mixture model \and Kullback-Leibler divergence \and human mobility.}




\end{document}
