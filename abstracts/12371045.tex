\documentclass[../booklet.tex]{subfiles}

\begin{document}

\subsection[Evidential Group Spammers Detection. {\it Malika Ben Khalifa, Zied Elouedi and Eric Lefevre}]{Evidential Group Spammers Detection}
\index[authors]{Khalifa, Malika Ben} \index[authors]{Elouedi, Zied} \index[authors]{Lefevre, Eric}

\begin{center}
  {\it Malika Ben Khalifa, Zied Elouedi and Eric Lefevre}
\end{center}
%\begin{minipage}{1\textwidth}
%\end{minipage}

%\vskip 0.8cm


\emph{}
Online reviews are considered as one of the most prevalent reference indicators for people to evaluate the quality of different products or services before purchasing. Since these reviews affect the buying decision of customers and control the success of the different e-commerce websites, the activity of fake reviews posting is more and more increasing. These fraudulent reviews are posted by a large number of spammers who try to promote or demote target products or companies. The reviewers spammers generally work collaboratively under group of spammers to take control of reviews given to some products, which seriously damage the review system. To deal with this issue, we propose a novel method aim to detect group spammers while relying on various group spamming behavioral indicators. Our approach is based on the K-nearest neighbors algorithm under the belief function theory to treat the uncertainty in the used behavioral indicators. Our method succeeds in distinguishing between genuine and fraudulent group of reviewers. It was tested on two large real datasets extracted from yelp.com.

\keywords{Fake reviews, Group spammers, Uncertainty, Belief Function Theory, Evidential KNN, E-commerce.}



\end{document}
