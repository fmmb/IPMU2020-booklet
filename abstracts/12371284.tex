\documentclass[../booklet.tex]{subfiles}

\begin{document}

\subsection[An undesirable behaviour of a recent extension of OWA operators to the setting of multidimensional data. {\it Raúl Pérez-Fernández}]{An undesirable behaviour of a recent extension of OWA operators to the setting of multidimensional data}
\index[authors]{Pérez-Fernández, Raúl}

\begin{center}
  {\it Raúl Pérez-Fernández}
\end{center}
%\begin{minipage}{1\textwidth}
%\end{minipage}

%\vskip 0.8cm

 % 15--250 words.

OWA operators have been ubiquitous in many disciplines since they were introduced by Yager in 1988. Aside of some other intuitive properties (e.g. monotonicity and idempotence), OWA operators are known to be continuous and, for some carefully constructed weighing vectors, very robust in the presence of outliers. In a recent paper, a natural extension of OWA operators to the setting of multidimensional data has been proposed based on the use of a linear extension of the product order by means of several weighted arithmetic means. Unfortunately, OWA operators constructed in such a way focus too strongly on the level sets of one of the weighted arithmetic means. It is here shown that this focus ultimately results in a forfeit of the properties of continuity and robustness in the presence of outliers. 

\keywords{OWA operator, Multidimensional data, Linear extension, Weighted arithmetic mean.} 



\end{document}
