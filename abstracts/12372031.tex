\documentclass[../booklet.tex]{subfiles}

\begin{document}

\section[A fuzzy approach for similarity measurement in time series, case study for stocks. {\it Soheyla Mirshahi and Vilem Novak}]{A fuzzy approach for similarity measurement in time series, case study for stocks}
\index[authors]{Mirshahi, Soheyla} \index[authors]{Novak, Vilem}

\begin{center}
  {\it Soheyla Mirshahi and Vilem Novak}
\end{center}
%\begin{minipage}{1\textwidth}
%\end{minipage}

\vskip 0.8cm


 
In this paper, we tackle the issue of assessing similarity among time series under the assumption that a time series can be additively decomposed into a trend-cycle and an irregular fluctuation.  It has been proved before that the former can be well estimated using the fuzzy transform. In the suggested method, first, we assign to each time series an adjoint one that consists of a sequence of trend-cycle of a time series estimated using fuzzy transform. Then we measure the distance between local trend-cycles. An experiment is conducted to demonstrate the advantages of the suggested method. This method is easy to calculate, well interpretable, and unlike standard euclidean distance, it is robust to outliers. 
\keywords{similarity measurements \and stock markets similarity \and time series analysis \and time series data mining.}



\end{document}
