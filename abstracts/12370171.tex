\documentclass[../booklet.tex]{subfiles}

\begin{document}

\subsection[Multi-agent systems and voting: how similar are voting procedures. {\it Janusz Kacprzyk, Jose M. Merigo, Hannu Nurmi and Slawomir Zadrozny}]{Multi-agent systems and voting: how similar are voting procedures}
\index[authors]{Kacprzyk, Janusz} \index[authors]{Merigo,  Jose M.} \index[authors]{Nurmi,  Hannu} \index[authors]{Zadrozny, Slawomir}

\begin{center}
  {\it Janusz Kacprzyk, Jose M. Merigo, Hannu Nurmi and Slawomir Zadrozny}
\end{center}
%\begin{minipage}{1\textwidth}
%\end{minipage}

\vskip 0.8cm


We consider the problem of the evaluation of similarity of voting procedures which are crucial in voting, social choice and related fields. We extend our approach 
proposed in our former works and compare the voting procedures against some well established and intuitively appealing 
criteria, and using the number of criteria satisfied as a point of departure for analysis. We also indicate potential of this approach for 
extending the setting to a fuzzy setting in which the criteria can be satisfied to a degree, and to include  a distance based analysis. A possibility to use elements of computational social choice is also indicated.\\[0.5cm]
\textbf{Keywords:} voting, social choice, voting procedure, similarity, binary pattern




\end{document}
