\documentclass[../booklet.tex]{subfiles}

\begin{document}

\subsection[Fuzzy inference system as an aggregation operator -    Application to the design of a soil chemical quality index. {\it Denys Yohana Mora-Herrera, Serge Guillaume, Didier Snoeck and Orlando Zuniga Escobar}]{Fuzzy inference system as an aggregation operator -    Application to the design of a soil chemical quality index}
\index[authors]{Mora-Herrera, Denys Yohana} \index[authors]{Guillaume, Serge} \index[authors]{Snoeck, Didier} \index[authors]{Escobar, Orlando Zuniga}

\begin{center}
  {\it Denys Yohana Mora-Herrera, Serge Guillaume, Didier Snoeck and Orlando Zuniga Escobar}
\end{center}
%\begin{minipage}{1\textwidth}
%\end{minipage}

\vskip 0.8cm


  Fuzzy logic is widely used in linguistic modeling. In this work, fuzzy logic is used in a multicriteria decision making framework in two different ways. First, fuzzy sets are used to model an expert preference relation for each of the individual information sources to turn raw data into satisfaction degrees. Second, fuzzy rules are used to model the interaction between sources to aggregate the individual degrees into a global score. The whole framework is implemented as an open source software called \textit{GeoFIS}. The potential of the method is illustrated using an agronomic case study to design a soil chemical quality index from expert knowledge for cacao production systems. The data come from three municipalities of Tolima department in Colombia. The output inferred by the fuzzy inference system was used as a target to learn the weights of classical numerical aggregation operators. Only the \textit{Choquet Integral} proved to have a similar modeling ability, but the weights would have been difficult to set from expert knowledge without learning.
  
\keywords{Fusion  \and  Multicriteria \and  Preference  \and Decision.}



\end{document}
