\documentclass[../booklet.tex]{subfiles}

\begin{document}

\subsection[On the logic of left-continuous t-norms and right-continuous t-conorms. {\it Lluis Godo, Martin Socola-Ramos and Francesc Esteva}]{On the logic of left-continuous t-norms and right-continuous t-conorms}
\index[authors]{Godo, Lluis} \index[authors]{Socola-Ramos,  Martin} \index[authors]{Esteva, Francesc}

\begin{center}
  {\it Lluis Godo, Martin Socola-Ramos and Francesc Esteva}
\end{center}
%\begin{minipage}{1\textwidth}
%\end{minipage}

\vskip 0.8cm


Double residuated lattices are expansions of residuated lattices with an extra monoidal operator, playing the role of a strong disjunction operation, together with its dual residuum. They were introduced by Or{\l}owska and Radzikowska. In this paper we consider the subclass of double residuated structures that are expansions  of MTL-algebras, that is, prelinear, bounded, commutative and integral residuated lattices. MTL-algebras constitute the algebraic semantics for the MTL logic, the system of mathematical fuzzy logic that is complete w.r.t. the class of residuated lattices on the real unit interval $[0,1]$ induced by left-continuous t-norms. Our aim is to axiomatise the logic whose intended semantics are commutative and integral double residuated structures on $[0, 1]$, that are induced by an arbitrary left-continuous t-norm, an arbitrary right-continuous t-conorm, and their corresponding residual operations.
\keywords{Mathematical fuzzy logic, Double residuated lattices, MTL, DMCTL, Semilinear logics, Standard completeness}



\end{document}
