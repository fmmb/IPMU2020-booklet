\documentclass[../booklet.tex]{subfiles}

\begin{document}

\subsection[Planning Wi-Fi access points activation in Havana City: proposal and preliminary results. {\it Cynthia Porras Nodarse, Jenny Fajardo Calderín, Alejandro Rosete Súarez and David A. Pelta}]{Planning Wi-Fi access points activation in Havana City: proposal and preliminary results}
\index[authors]{Nodarse, Cynthia Porras} \index[authors]{Calderín, Jenny Fajardo} \index[authors]{Súarez, Alejandro Rosete} \index[authors]{Pelta, David A.}

\begin{center}
  {\it Cynthia Porras Nodarse, Jenny Fajardo Calderín, Alejandro Rosete Súarez and David A. Pelta}
\end{center}
%\begin{minipage}{1\textwidth}
%\end{minipage}

\vskip 0.8cm


The availability of Wi-Fi connection points or hotspots in places such as parks, transport stations, libraries, and so on is one of the key aspects to allow people the usage of Internet resources (to study, work or meet).
This is even more important in Central America and Caribbean countries where the deployment of huge cost infrastructure (like optical fiber) to provide Internet access at home is not envisaged neither in the short or mid term. And this is clearly the case in Havana, Cuba. 

This contribution presents the problem of planning the Wi-Fi access points activation, where each point can have different signal power levels and availability along the time. Due to power consumption constraints, it is impossible to have all the points activated simultaneously with maximum signal strength.

The problem is modelled as a dynamic maximal covering location one with facility types and time dependant availability. A metaheuristic approach is used to solve the problem by using an Algorithm portfolio and examples on how solutions can be analyzed (beyond the coverage provided)  are shown.

\keywords{signal levels \and Wi-Fi access points.}



\end{document}
