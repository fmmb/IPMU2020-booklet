\documentclass[../booklet.tex]{subfiles}

\begin{document}

\section[An Initial Study on Typical Hesitant (T,N)-Implication Functions. {\it Mônica Lorea Matzenauer, Renata Reiser, Helida Santos, Jocivania Pinheiro and Benjamin Bedregal}]{An Initial Study on Typical Hesitant (T,N)-Implication Functions}
\index[authors]{Matzenauer, Mônica Lorea} \index[authors]{Reiser,  Renata} \index[authors]{Santos,  Helida} \index[authors]{Pinheiro,  Jocivania} \index[authors]{Bedregal, Benjamin}

\begin{center}
  {\it Mônica Lorea Matzenauer, Renata Reiser, Helida Santos, Jocivania Pinheiro and Benjamin Bedregal}
\end{center}
%\begin{minipage}{1\textwidth}
%\end{minipage}

\vskip 0.8cm


In the theory of Hesitant Fuzzy Sets (HFS), the membership degree of an element is characterized by a membership function which always returns a fuzzy set. This approach enables one to express, for example, the hesitance of several experts in the process of decision making based on multiple attributes and multiple criteria. In this work, we focus on the study of a class of implication functions for typical hesitant fuzzy sets (THFS). The novelty of our proposal lies on the fact that it is the first time that an admissible order is used to define operators on hesitant fuzzy setting. Thus, we introduce  typical hesitant fuzzy negations, typical hesitant t-norms and  typical hesitant implication functions considering an admissible order, which allows the comparison of typical hesitant fuzzy elements with different cardinalities.


\keywords{Hesitant Fuzzy Sets \and Admissible Orders on THFS \and Typical Hesitant Implication Functions  \and (T,N)-Implication Functions.}



\end{document}
