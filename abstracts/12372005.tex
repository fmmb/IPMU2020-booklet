\documentclass[../booklet.tex]{subfiles}

\begin{document}

\subsection[Fuzzy Temporal Graphs and Sequence Modelling in  Scheduling Problem. {\it Margarita Knyazeva, Alexander Bozhenyuk and Uzay Kaymak}]{Fuzzy Temporal Graphs and Sequence Modelling in  Scheduling Problem}
\index[authors]{Knyazeva, Margarita} \index[authors]{Bozhenyuk, Alexander} \index[authors]{Kaymak, Uzay}

\begin{center}
  {\it Margarita Knyazeva, Alexander Bozhenyuk and Uzay Kaymak}
\end{center}
%\begin{minipage}{1\textwidth}
%\end{minipage}

%\vskip 0.8cm

Processing sequential data and time-dependent data is a problem of constructing computational graph with a certain structure. A computational graph formalizes the structure of a set of computations including mapping temporal inputs and outputs. In this paper we apply graph theory and fuzzy interval representation of uncertain variables to indicate states of the temporal scheduling system. Descriptive model for temporal reasoning on graph, sequence modelling and ordering of fuzzy inputs for scheduling problem is introduced.

\keywords{Fuzzy Sequence Modelling, Computational Graph, Fuzzy Graph, Fuzzy Temporal Intervals, Temporal Reasoning, State-Transition System}



\end{document}
