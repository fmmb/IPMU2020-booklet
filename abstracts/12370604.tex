\documentclass[../booklet.tex]{subfiles}

\begin{document}

\section[Automatic detection of symmetry in dermoscopic images based on shape and texture. {\it Vincent Toureau, Pedro Bibiloni, Lidia Talavera-Martínez and Manuel González-Hidalgo}]{Automatic detection of symmetry in dermoscopic images based on shape and texture}
\index[authors]{Toureau, Vincent} \index[authors]{Bibiloni,  Pedro} \index[authors]{Talavera-Martínez,  Lidia} \index[authors]{González-Hidalgo, Manuel}

\begin{center}
  {\it Vincent Toureau, Pedro Bibiloni, Lidia Talavera-Martínez and Manuel González-Hidalgo}
\end{center}
%\begin{minipage}{1\textwidth}
%\end{minipage}

\vskip 0.8cm


% One-liner
In this paper we present computational methods to detect the symmetry in dermoscopic images of skin lesions.
% Motivation
Skin lesions are assessed by dermatologists based on a number of factors.
In the literature, the asymmetry of lesions appears recurrently since it may indicate irregular growth.
% What we want
We aim at developing an automatic algorithm that can detect symmetry in skin lesions, as well as indicating the axes of symmetry.
% What we accomplish
We tackle this task based on skin lesions' shape, based on their color and texture, and based on their combination.
To do so, we consider symmetry axes through the center of mass, random forests classifiers to aggregate across different orientations, and a purposely-built dataset to compare textures that are specific of dermoscopic imagery.
We obtain 84-88\% accuracy in comparison with samples manually labeled as having either 1-axis symmetry, 2-axes symmetry or as being asymmetric.
% Final thoughts
Besides its diagnostic value, the symmetry of a lesion also explains the reasons that might support such diagnosis.
Our algorithm does so by indicating how many axes of symmetry were found, and by explicitly computing them.

\keywords{dermoscopic images\and skin lesion \and computational methods \and symmetry detection\and shape \and texture \and color\and machine learning \and random forest}



\end{document}
