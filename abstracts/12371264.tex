\documentclass[../booklet.tex]{subfiles}

\begin{document}

\subsection[Dissimilarity based Choquet integrals. {\it Humberto Bustince, Radko Mesiar, Javier Fernandez, Mikel Galar, Daniel Paternain, Abdulrahman Altalhi, Gracaliz Dimuro, Benjamin Bedregal and Zdenko Takac}]{Dissimilarity based Choquet integrals}
\index[authors]{Bustince, Humberto} \index[authors]{Mesiar, Radko} \index[authors]{Fernandez, Javier} \index[authors]{Galar, Mikel} \index[authors]{Paternain, Daniel} \index[authors]{Altalhi, Abdulrahman} \index[authors]{Dimuro, Gracaliz} \index[authors]{Bedregal, Benjamin} \index[authors]{Takac, Zdenko}

\begin{center}
  {\it Humberto Bustince, Radko Mesiar, Javier Fernandez, Mikel Galar, Daniel Paternain, Abdulrahman Altalhi, Gracaliz Dimuro, Benjamin Bedregal and Zdenko Takac}
\end{center}
%\begin{minipage}{1\textwidth}
%\end{minipage}

%\vskip 0.8cm


In this paper, in order to generalize the Choquet integral, we replace the difference between inputs in its definition by a restricted dissimilarity function and refer to the obtained function as $d$-Choquet integral. For some particular restricted dissimilarity function the corresponding $d$-Choquet integral with respect to a fuzzy measure is just the 'standard' Choquet integral with respect to the same fuzzy measure. Hence, the class of all $d$-Choquet integrals encompasses the class of all 'standard' Choquet integrals. This approach allows us to construct a wide class of new functions, $d$-Choquet integrals, that are possibly, unlike the 'standard' Choquet integral, outside of the scope of aggregation functions since the monotonicity is, for some restricted dissimilarity function, violated and also the range of such functions can be wider than $[0,1]$, in particular it can be $[0,n]$.

\keywords{Choquet integral, d-Choquet integral, dissimilarity, pre-aggregation function, aggregation function, monotonicity, directional monotonicity.}



\end{document}
