\documentclass[../booklet.tex]{subfiles}

\begin{document}

\subsection[Explaining the neural network: A case study to model the incidence of cervical cancer. {\it Paulo Lisboa, Sandra Ortega and Ivan Olier}]{Explaining the neural network: A case study to model the incidence of cervical cancer}
\index[authors]{Lisboa, Paulo} \index[authors]{Ortega,  Sandra} \index[authors]{Olier, Ivan}

\begin{center}
  {\it Paulo Lisboa, Sandra Ortega and Ivan Olier}
\end{center}
%\begin{minipage}{1\textwidth}
%\end{minipage}

\vskip 0.8cm

Neural networks are frequently applied to medical data. We describe how complex and imbalanced data can be modelled with simple but accurate neural networks that are transparent to the user. In the case of a data set on cervical cancer with 753 observations excluding, missing values, and 32 covariates, with a prevalence of 73 cases (9.69%), we explain how model selection can be applied to the Multi-Layer Perceptron (MLP) by deriving a representation using a General Additive Neural Network. 

     The model achieves an AUROC of 0.621 CI [0.519,0.721] for predicting positive diagnosis with Schiller's test.  This is comparable with the perfor-mance obtained by a deep learning network with an AUROC of 0.667 [1]. Instead of using all covariates, the Partial Response Network (PRN) in-volves just 2 variables, namely the number of years on Hormonal Contra-ceptives and the number of years using IUD, in a fully explained model. This is consistent with an additive non-linear statistical approach, the Sparse Additive Model [2] which estimates non-linear components in a lo-gistic regression classifier using the backfitting algorithm applied to an ANOVA functional expansion.

     This paper shows how the PRN, applied to a challenging classification task, can provide insights into the influential variables, in this case corre-lated with incidence of cervical cancer, so reducing the number of unnecessary variables to be collected for screening. It does so by exploiting the effi-ciency of sparse statistical models to select features from an ANOVA de-composition of the MLP, in the process deriving a fully interpretable model

\keywords{Explainable Machine Learning, FATE, KDD, Medical Decision Support, Cervical Cancer}


\end{document}
