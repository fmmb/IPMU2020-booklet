
The discovery of knowledge by analyzing time series is an important field of research. In this paper we investigate multiple multivariate time series, because we assume a higher information value than regarding only one time series at a time. There are several approaches which make use of the granger causality or the cross correlation in order to analyze the influence of time series on each other. In this paper we extend the idea of mutual influence and present FCSETS (\textbf{\underline{F}}uzzy \textbf{\underline{C}}lustering \textbf{\underline{S}}tability \textbf{\underline{E}}valuation of \textbf{\underline{T}}ime \textbf{\underline{S}}eries), a new approach which makes use of the membership degree produced by the fuzzy c-means (FCM) algorithm. We first cluster time series per timestamp and then compare the relative assignment agreement (introduced by Eyke H{\"u}ller\-meier and Maria Rifqi) of all subsequences. This leads us to a stability score for every time series which itself can be used to evaluate single time series in the data set. It is then used to rate the stability of the entire clustering. The stability score of a time series is higher the more the time series sticks to its peers over time. This not only reveals a new idea of mutual time series impact but also enables the identification of an optimal amount of clusters per timestamp.
We applied our model on different data, such as financial, country related economy and generated data, and present the results.

\keywords{Time Series Analysis \and Fuzzy Clustering \and Evaluation}

