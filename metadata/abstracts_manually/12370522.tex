
Analogical proportions, often denoted $A:B::C:D$,  are statements of the form ``$A$ is to $B$ as $C$ is to $D$'' that involve comparisons between items. They are at the basis of an inference mechanism that has been recognized as a suitable tool for classification and has led to a variety of analogical classifiers in the last decade. Given an object $D$ to be classified, the basic idea of such classifiers is to look
for triples of examples $(A, B, C)$, in the learning set, that form an analogical proportion with D, on a maximum set of attributes. In the context of classification, objects $A, B, C$ and $D$ are assumed to be represented by vectors of feature values. Analogical inference relies on the fact that if a proportion $A:B::C:D$ is valid, one of the four components of the proportion can be computed from the three others. Based on this principle, analogical classifiers have a cubic complexity due to the search for all possible triples in a learning set to make a single prediction.
A special case of analogical proportions involving only three items $A, B$ and $C$ are called \textit{continuous} analogical proportions and are of the form ``$A$ is to $B$ as $B$ is to $C$'' (hence  denoted $A:B::B:C$). In this paper, we develop a new classification algorithm based on continuous analogical proportions and applied to numerical features. Focusing on pairs rather than triples, the proposed classifier enables us to compute an unknown midpoint item $B$ given a pair of items $(A,C)$. Experimental results of such classifier show an efficiency close to the previous analogy-based classifier while maintaining a reduced quadratic complexity.

\keywords{Classification, Analogical proportions, Continuous analogical proportions}

