
In many practical situations, we only know the interval containing
the quantity of interest, we have no information about the
probabilities of different values within this interval. In contrast
to the cases when we know the distributions and can thus use
Monte-Carlo simulations, processing such interval uncertainty is
difficult -- crudely speaking, because we need to try all possible
distributions on this interval. Sometimes, the problem can be
simplified: namely, for estimating the range of values of some characteristics of the distribution,
it is possible to select a single distribution
(or a small family of distributions) whose analysis provides a good
understanding of the situation. The most known case is when we are estimating the largest possible of Shannon's entropy: in this case, it is sufficient to consider the uniform distribution on
the interval. Interesting, estimating other characteristics leads to the selection of the same
uniform distribution: e.g., estimating the largest possible values of generalized entropy or of some sensitivity-related characteristics. 
In this paper, we provide a general
explanation of why uniform distribution appears in different
situations -- namely, it appears every time we have a
permutation-invariant optimization problem with the unique optimum.
We also discuss what happens if we have an optimization problem that attains its optimum at several 
different distributions -- this happens, e.g., when we are estimating the smallest possible 
value of Shannon's entropy (or of its generalizations).
\keywords{Interval uncertainty \and Maximum Entropy approach \and
Uniform distribution \and Sensitivity analysis}

