
Statistical image analysis is an extensive field that includes problems such as noise-reduction, de-blurring, feature enhancement, and object detection/identification, to name a few. Bayesian image analysis can improve image quality, by balancing a priori expectations of image characteristics, with a model for the noise process via Bayes Theorem. We have previously given a reformulation of the conventional Bayesian image analysis paradigm in Fourier space, i.e. the prior distribution (the prior) and likelihood are given in terms of spatial frequency signals. By specifying the Bayesian model in Fourier space, spatially correlated priors, that are relatively difficult to model and compute in conventional image space, can be efficiently modeled as a set of independent processes across Fourier space. The originally inter-correlated and high-dimensional problem in image space is thereby broken down into a series of (trivially parallelizable) independent one-dimensional problems. In this paper we adapt this Fourier space process into a data-driven framework in which the Fourier space priors are built empirically from a database of images and then used to enhance future images. We will describe the data-driven Bayesian image analysis in Fourier space (DD-BIFS) modeling approach, illustrate it's computational efficiency and speed. Finally, we give specific applications of DD-BIFS to improve the quality of arterial-spin-labeling (ASL) perfusion images via a database of human brain positron emission tomography (PET) images.

\keywords{Bayesian image analysis \and data-driven priors \and Fourier space.}

