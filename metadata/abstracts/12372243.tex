
In the processing of imprecise information, principally in big data analysis, it is very advantageous to transform numerical values into the standard form of linguistic statements. 
This paper deals with a novel method of outlier detection using linguistic summaries. Particular attention is devoted to examining the usefulness of non-monotonic quantifiers, which represent a fuzzy determination of the amount of analyzed data.The answer is positive. The use of non-monotonic quantifiers in the detection of outliers can provide a more significant value of the degree of truth of a linguistic summary. At the end, this paper provides a computational example of practical importance.

\keywords{intelligent data analysis, linguistic summaries, monotonic and non-monotonic quantifiers, intelligent outlier detection}

