Bayesian and probabilistic models are widely used in image processing to handle noise due to various alteration phenomena. To benefit from the spatial information in a tractable way, Markov Random Fields (MRF) are often assumed with isotropic neighborhoods, that is however at the
detriment of the preservation of thin structures. In this study, we aim at relaxing this assumption on stationarity and isotropy of the neighborhood shape in order to get a prior probability term that is relevant not only within the homogeneous areas but also close to object borders and within thin structures. To tackle the issue of neighborhood shape estimation, we propose to use tensor voting, that allows for the estimation of structure direction and saliency at various scales. We propose three main ways to derive anisotropic neighborhoods, namely shape-based, target-based and cardinal-based neighborhood. Then, having defined the neighborhood field, we introduce an
energy that will be minimized using graph cuts, and illustrate the benefits of our approach against the use of isotropic neighborhoods in the applicative context of crack detection. First results on such a challenging problem are very encouraging.

\keywords{Thin structures \and Segmentation \and Anisotropic neighborhoods \and Superpixels \and Graph cuts.}
