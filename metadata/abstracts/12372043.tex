
Time series are temporal ordered data available in many fields of science such as medicine, physics, astronomy, audio, etc. Various methods have been proposed to analyze time series. Amongst them, time series classification consists in predicting the class of a time series according to a set of already classified data. However, the performance of a time series classification algorithm depends on the quality of the known labels.  In real applications, time series are often labeled by an expert or by an imprecise process, leading to noisy classes. Several algorithms have been developed to handle uncertain labels in case of non-temporal data sets. As an example, the fuzzy k-NN introduces for labeled objects a degree of membership to belong to classes. In this paper, we combine two popular time series classification algorithms, Bag of SFA Symbols (BOSS) and the Dynamic Time Warping (DTW) with the fuzzy k-NN. The new algorithms are called Fuzzy DTW and Fuzzy BOSS. Results show that our fuzzy time series classification algorithms outperform the non-soft algorithms especially when the level of noise is high.

%Time series are used in many fields of science such as medicine, physics, astronomy, audio, etc. Several methods were specially developed to manage with the constraint of temporal ordering of the data. They can predict future values, cluster time series, detect anomalies or more specifically classify. Time Series Classification (TSC) consists in attributing a label of time series according to a training set. The problem is that the performance of the classification depends on the well labelling of the series in the training set.  However, the time series are often labelled by an expert or by an imprecise process. This creates noisy labels that reduces the classification accuracy. Several algorithms were developed to handle the noisy labels, they are called soft algorithms. In this paper, we convert two popular TSC algorithms, Bog of SFA Symbols (BOSS) and the Dynamic Time Warping (DTW) algorithms that we test with the soft k-NN algorithm. The results show that the soft TSC algorithms outperform the non-soft algorithms especially when the level of noise is high. The results also show that the soft BOSS and DTW are more appropriated than the soft k-NN algorithm.  
%The abstract should briefly summarize the contents of the paper in 150--250 words. 6 to 14 pages IPMU
\keywords{time series classification \and BOSS \and fuzzy k-NN \and soft labels.}

