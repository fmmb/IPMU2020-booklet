Dealing with uncertainty has always been a challenging topic in the area of knowledge representation. Nowadays, as the internet provides a vast platform for knowledge exploitation, the need becomes even more imminent. The kind of uncertainty encountered in most of these cases as well as its distributed nature make Dempster-Shafer (D-S) Theory to be an appropriate framework for its representation. However, we have to face the drawback of the computation burden of Dempster's rule of combination due to its combinatorial behavior. Constraint Programming (CP) has proved to be an efficient tool in cases where results have to satisfy some specified properties and pruning of the computation space can be achieved. As D-S theory measures' computation fulfills this requirement, CP seems a promising framework to employ for this purpose. In this paper, we present our approach to use CP to compute the belief and plausibility measures of D-S Theory and Dempster's rule of combination as well as the results of the effort. As it was expected, the results are quite promising and in many cases impressive.

\keywords{Dempster-Shafer Theory, Uncertainty, Constraint Programming, ECLiPSe Prolog}


