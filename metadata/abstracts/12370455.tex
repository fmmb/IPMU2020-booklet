
Supervised learning is an important branch of machine learning (ML), which requires a complete annotation (labeling) of the involved training data. This assumption, which may constitute a severe bottleneck in the practical use of ML, is relaxed in weakly supervised learning. In this ML paradigm, training instances are not necessarily precisely labeled. Instead, annotations are allowed to be imprecise or partial. In the setting of superset learning,  instances are assumed to be labeled with a set of \emph{possible} annotations, which is assumed to contain the correct one.
In this article, we study the application of \emph{rough set theory} in the setting of superset learning. In particular, we consider the problem of feature reduction as a mean for \emph{data disambiguation}, i.e., for the purpose of figuring out the most plausible precise instantiation of the imprecise training data.
 To this end, we define appropriate generalizations of decision tables and reducts, using information-theoretic techniques based on evidence theory. Moreover, we analyze the complexity of the associated computational problems.
\keywords{Feature Selection, Superset Learning, Rough Sets, Evidence Theory.}

