
In this paper, imprecise approaches to model the risk reserve process of an insurer's portfolio, which consists of a catastrophe bond and external help, and with a special penalty function in the case of a bankruptcy event, are presented.
Apart from the general framework, two special cases, when parameters of the portfolio are described by L-R fuzzy numbers or shadowed sets, are discussed and compared.
In a few examples based on the real-life data for these two types of impreciseness, some important characteristics of the portfolio, like the expected value and the probability of the ruin, are estimated, analysed and compared using the Monte Carlo simulations.  

\keywords{Risk process  \and Fuzzy numbers \and Shadowed sets \and Insurance portfolio \and Numerical simulations.}

