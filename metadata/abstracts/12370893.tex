
I introduce and study a new notion of Archimedeanity for binary and non-binary choice between options that live in an abstract Banach space, through a very general class of choice models, called sets of desirable option sets.
In order to be able to bring horse lottery options into the fold, I pay special attention to the case where these linear spaces do not include all `constant' options.
I consider the frameworks of conservative inference associated with Archimedean choice models, and also pay quite a lot of attention to representation of general (non-binary) choice models in terms of the simpler, binary ones.
The representation theorems proved here provide an axiomatic characterisation of, amongst other choice methods, Levi's E-admissibility and Walley--Sen maximality.

\keywords{Choice function \and coherence \and {\archimty} \and set of desirable option sets.}

