
Power quality analysis involves the measurement of quantities that characterize a power supply waveform such as its frequency. The measurement of those quantities are regulated by internationally accepted standards from IEEE or IEC. Monitoring the delivered power quality is even more important due to recent advances in power electronics and also due to the increasing penetration of renewable energies in the electrical power grid. The primary suggested method by IEC to measure the power grid frequency is to count the number of zero crossings in the voltage waveform that occur during 0.2~s. The standard zero crossing method is usually applied to a filtered signal that has a non deterministic and frequency dependent delay. For monitoring the power grid a range between 42.5 and 57.5~Hz should be considered which means that the filter must be designed in order to attenuate the delay compensation error. Fuzzy Boolean Nets can be considered a neural fuzzy model where the fuzziness is an inherent emerging property that can ignore some outliers acting as a filter. This property can be useful to apply zero crossing without false crossing detection and estimate the real timestamp without the non deterministic delay concern. This paper presents a comparison between the standard frequency estimation, a Goertzel interpolation method, and the standard method applied after a FBN network instead of a filtered signal.

\keywords{Power quality \and Frequency estimation \and Zero crossing \and Neural Networks \and FBN.}

