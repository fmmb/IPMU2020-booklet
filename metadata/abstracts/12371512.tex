
Uncertain data are observations that cannot be uniquely mapped to
a referent. In the case of uncertainty due to incompleteness, possibility
theory can be used as an appropriate model for processing such data.
In particular, granular counting is a way to count data in presence
of uncertainty represented by possibility distributions. Two algorithms
were proposed in literature to compute granular counting: exact granular
counting, with quadratic time complexity, and approximate granular
counting, with linear time complexity. This paper extends approximate
granular counting by computing bounds for exact granular count. In
this way, the efficiency of approximate granular count is combined
with certified bounds whose width can be adjusted in accordance to
user needs.\keywords{Granular Counting \and Possibility Theory \and Uncertain Data}.

